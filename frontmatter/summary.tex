\begin{center}
    {\large \textbf{SUMMARY}}
\end{center}

\thispagestyle{plain}

One of the foremost concerns today is the growing inequality in nearly all aspects of life and its implications for individuals' subjective well-being. This dissertation how individuals and their family members' well-being is impacted by three important family-related and relationship-oriented factors: intergenerational mobility, educational sorting in unions, and trust.

Specifically, Chapter~\ref{chap:edu-mobility-swb} addresses the well-being outcome of intergenerational educational mobility, for Chinese families. It is the first to encompass both adult children and their parents, with attention to family structures and parent-child gender dynamics. Employing the Diagonal Mobility Model (DMM) to analyze data from the China Family Panel Studies, I found that net of origin and destination, those moving upwards were not necessarily better off, but both generations' well-being was impaired by downward mobility—a phenomenon observed exclusively among only-child families. Among these parents, mothers with an upwardly mobile daughter reported the highest life satisfaction. These findings highlight the departure from traditional son preferences and the plight of mobility—penalties for falling downwards with no returns to moving upwards—that traps both generations in only-child Chinese families.

Chapter~\ref{chap:edu-sorting-swb} investigates the association between educational sorting and subjective well-being of heterosexual partners in Europe. It extends the literature by exploring how these associations differ between genders and vary across societies and normative climates. Using the DMM to analyze data from the European Social Survey, I found that net of status effects, hypergamy (women partnering with more educated men) was associated with lower well-being for both genders, and men were more satisfied with life in hypogamy (partnering with more educated women). Notably, women's well-being disadvantage in hypergamy was exacerbated in contexts where such partnerships were less normative. In other words, hypergamy may become even more unfavorable for women over time, as a further decline in hypergamy is likely in the foreseeable future.

Chapter~\ref{chap:trust-swb} recognizes trust as a critical determinant of not only individuals' but also their spouse's subjective well-being across multiple dimensions: life satisfaction, happiness, and depression. A sample of 6,964 pairs of married couples was selected from the China Family Panel Studies in 2014 and 2018 and analyzed using the Actor-Partner Interdependence Models with Mediation. Results show that particularized trust, through improving marital satisfaction and interpersonal relationships, significantly enhanced individuals' well-being in all dimensions and extended its benefits from husbands to wives, but not vice versa. Generalized trust had limited effects and did not spill over. These findings highlight the gendered dynamics within husband-wife dyadic interactions and underscore the importance of relational contexts in shaping the well-being implications of trust.

This dissertation underscores the crucial role of relationships and by extension, the perspective of linked lives, in understanding families' well-being in the context of fast-changing demographic and stratification patterns. Future research should explore the emergence of non-conventional families and how families, in turn, shape stratification.

\clearpage