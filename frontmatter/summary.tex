\begin{center}
    {\large \textbf{SUMMARY}}
\end{center}

\thispagestyle{plain}

One of the foremost concerns today is the growing inequality in nearly all aspects of life. This dissertation on stratification, family, and well-being examines the changing patterns of family formation and how individuals and their family members are impacted by family-related stratification-generating mechanisms (e.g., intergenerational mobility, mate selection).

Specifically, Chapter 2 addresses the well-being outcome of intergenerational educational mobility, for Chinese families. It is the first to encompass both adult children and their parents, with attention to family structures and parent-child gender dynamics. Employing the Diagonal Mobility Model (DMM) to analyze data from the China Family Panel Studies, I found that net of origin and destination, those moving upwards were not necessarily better off, but both generations' well-being was impaired by downward mobility—a phenomenon observed exclusively among only-child families. Among these parents, mothers with an upwardly mobile daughter reported the highest life satisfaction. These findings highlight the departure from traditional son preferences and the plight of mobility—penalties for falling downwards with no returns to moving upwards—that traps both generations in only-child Chinese families.

Chapter 3 investigates the association between educational sorting-the spousal difference in education-and subjective well-being of heterosexual partners in Europe. It extends the literature by exploring how these associations differ between genders and vary across societies and normative climates. Using the DMM to analyze data from the European Social Survey, I found that net of status effects, hypergamy (women partnering with more educated men) was associated with lower well-being for both genders, and men were more satisfied with life in hypogamy (partnering with more educated women). Notably, women's well-being disadvantage in hypergamy was exacerbated in contexts where such partnerships were less normative. In other words, hypergamy may become even more unfavorable for women over time, as a further decline in hypergamy is likely in the foreseeable future.

Chapter 4 documents and explains the trends in marital sorting by education in China, using a decomposition approach to unpack the specific contributions of educational expansion, educational gradients in marriage rates, and assortative mating preferences. Results indicate that the initial decrease in homogamy among cohorts born before 1965 were driven entirely by educational expansion. For later cohorts, sustained educational expansion became unfavorable for hypergamy, outweighing the opposing effects of a steeper decline in marriage rates among highly educated women. Preferences for homogamy and against heterogamy, especially hypogamy, intensified. Together, these factors explained the rising homogamy, the declining hypergamy, the stagnant hypogamy, as well as the converging urban-rural disparities across later cohorts.

This dissertation contributes to a more nuanced understanding of the experience of families in the context of stratification. Future research should explore the emergence of non-conventional family forms and the broader implications of stratification.

\clearpage