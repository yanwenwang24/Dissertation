\section{Theoretical Framework}
\label{sec:ch4-theoretical-framework}

\subsection{Trust in the Chinese Context}

\subsubsection{Conceptualizing Trust: Generalized and Particularized Trust}

Trust is a multidimensional construct with varying meanings across contexts. As \textcite{barberLogicLimitsTrust1983}noted, the meanings of trust are socially learned and cannot be isolated from the social environment and culture within a specific context. Therefore, it is crucial to clarify the types of trust considered in this study before examining their well-being implications.

In Chinese society characterized by strong familism, trust is viewed as being intrinsically linked to and strictly confined within familial relationships and kinship networks \parencite{weberReligionChinaConfucianism1968,yanIntergenerationalIntimacyDescending2016,yanChineseFamiliesUpside2021}. Therefore, this type of trust cannot be generalized to those outside blood-tied communities. Some scholars, like \textcite{fukuyamaTrustSocialVirtues1996}, have argued that familial societies, such as China and Italy, lack the soil for fostering trust towards strangers and the general population.

However, this narrow view of strictly kinship-based trust has been challenged. \textcite{feiSoilFoundationsChinese1992}, in his seminal work From the Soil, introduced the "patterns of difference (\textit{chaxugeju})," likening interpersonal relationships (\textit{guanxi}) in China to ripples forming and spreading across the surface of a lake. As the ripple moves further outward, it diminishes and eventually becomes insignificant to the individual at the center. This analogy vividly illustrates that trust may extend beyond kinship networks through interpersonal relationships, though it wanes and eventually dissolves as it approaches the "trust radius"—the maximum distance between the subject and objects of trust \parencite{liParticularizedTrustGeneralized2002}.

Over recent decades, China's marketization and modernization have somewhat diminished familism and fostered socioeconomic conditions supportive of trust towards the general population \parencite{xuChangesMainlandChinese2014,yanChineseFamiliesUpside2021}. Dramatic social changes, including rapid urbanization, educational expansion, and the emerging civil sector, have broadened individuals' interactions between their immediate social circles \parencite{bianOccupationClassSocial2005,steinhardtSocioEconomicModernizationCrisis2020}. These social changes have catalyzed a modern society, where the development of a moralized, unconditional trust in others becomes possible.

Against this backdrop, this study distinguishes two types of trust in China. On one end of the spectrum is particularized trust on known, familiar individuals within one's immediate social circles \parencite{delheyHowGeneralTrust2011,kramerIngroupOutgroupTrustBarriers2018,steinhardtSocioEconomicModernizationCrisis2020}. This form of trust is confined within the limits of interpersonal relationships and rarely extends beyond the trust radius. At the other end lies generalized trust, also known as thick, diffuse, or moralized trust, which reflects an overarching belief in the "the benevolence of human nature \parencite[p.~139]{yamagishiTrustCommitmentUnited1994}." Unlike particularized trust, generalized trust is not reliant on personal interactions and encompasses strangers and the general population.

\subsubsection{Measuring Generalized and Particularized Trust}

Generalized trust is often measured by the standard survey question: "Generally speaking, would you say that most people can be trusted or that you need to be very careful in dealing with people?" \parencite{rosenbergMisanthropyPoliticalIdeology1956}. Adopted from the American General Social Surveys to many cross-national surveys like the World Value Survey (WVS) and the European Social Survey (ESS), this survey item has gained popularity for it has a very low non-response rate and remains remarkably consistent over time within the same country \parencite{nannestadWhatHaveWe2008}. However, its validity and accuracy in non-Western contexts, especially in East Asia, have been increasingly challenged \parencite{delheyPredictingCrossNationalLevels2005,glaeserMeasuringTrust2000,torpeIdentifyingSocialTrust2011}. Given the ambiguity of "most people," respondents must rely on their own interpretations and give responses to whether they trust individuals within their "trust radius." Consequently, when the trust radius is narrow, the survey question may incorrectly identify particularized trust as generalized trust, making this measure difficult to compare across contexts \parencite{delheyHowGeneralTrust2011,delheyPredictingCrossNationalLevels2005,torpeIdentifyingSocialTrust2011}.

The trust radius in China remains relatively narrow. An earlier study found that the subjects of "most people" rarely extended beyond the immediate circles of contacts, such as family members, neighbors, and intimate friends \parencite{liParticularizedTrustGeneralized2002}. More recent cross-national studies calculated the trust radius as the correlations between trust toward specified in-group and out-group members, ranking China 47th out of 51 countries surveyed \parencite{delheyHowGeneralTrust2011}. These findings suggest that the standard survey item, with unspecified targets of trust, may not effectively capture the nuances of trust dynamics within the Chinese context.

Considering these concerns, it is important to use alternative survey items sensitive to the trust radius when measuring generalized and particularized trust. This study employs survey items that clearly define the targets of trust to ensure more accurate measures: individuals within close relationships (e.g., family members, neighbors) for particularized trust and those beyond the trust radius (e.g., strangers) for generalized trust.

\subsection{Trust and Couples' Subjective Well-Being}

\subsubsection{Trust as a Strong Determinant of Individual's Well-Being}

Generalized trust, when measured towards the unspecific "most people," is often regarded as one of the strongest determinant of individuals' subjective well-being \parencite{adedejiExaminingPathwaysGeneral2023,helliwellHowsLifeCombining2003}. However, when measured towards strangers or out-group members, findings appear to be more contentious. Several studies reported that individuals with higher levels of trust in strangers were happier, more satisfied with life, and exhibit fewer depressive symptoms \parencite{churchillTrustSocialNetworks2017,baiSocialTrustPattern2019}, whereas other research found no significant associations between trust towards out-group members and self-rated health in China, Japan, and Korea \parencite{sungTrustHealthCrossnational2019,sungIngroupTrustSelfrated2020}. One study in South Africa suggested that trust in strangers did not yield benefits, but instead led to worsened depressive symptoms \parencite{adjaye-gbewonyoHighSocialTrust2018}.

Empirical evidence on the well-being implications of particularized trust is more consistent across contexts. Utilizing data from the WVS, \textcite{sungTrustHealthCrossnational2019} found almost universal positive associations between trust towards in-group members and self-rated health worldwide. In East Asia, \textcite{sungIngroupTrustSelfrated2020} observed benefits of trust towards in-group members in all four regions surveyed (China, Korea, Japan, and Taiwan), whereas the positive influence of out-group trust on health was significant only in Taiwan. However, exceptions do exist. For instance, \textcite{baiSocialTrustPattern2019} found that in China, individuals' happiness was positively associated with generalized trust toward strangers but unaffected by trust in family members.
Considering these findings, I propose the following hypotheses at the individual level.

\begin{quote}
    \textbf{H1a.} Generalized trust is positively associated with individuals' life satisfaction and happiness, and negatively associated with depression.

    \textbf{H1b.} Particularized trust is positively associated with individuals' life satisfaction and happiness, and negatively associated with depression.
\end{quote}

\subsubsection{Contextualizing Trust in Marriage: The Spillover Effects of Trust}

Despite abundant evidence on the individual- and contextual-level outcomes of trust, there has been limited research on how trust functions within relationships, specifically, how one person's trust may extend its benefits to others via relational pathways. The current study shifts the perspective from viewing trust as merely an individualistic attribute to emphasizing its relational nature and contextualizing it within dyadic interactions. Specifically, I focus on marital relationships, in which individuals' trust may not only influence their own subjective well-being but also spill over to impact their spouse's.

I focus on husband-wife dyads for several reasons. First, marriage remains one of the most prevalent and salient relationships in China \parencite{jiangMarriageSqueezeNeverMarried2014,yanChineseFamiliesUpside2021}. Despite declining marriage rates, China is still characterized by universal marriage and traditional familial values \parencite{jiHeterogeneityContemporaryChinese2014}. Fewer than 5\% of women and 10\% of men remained single by the age of 30 to 34, and the percentage of the lifelong single population has consistently stayed below 5\% \parencite{jiangMarriageSqueezeNeverMarried2014}. Therefore, focusing on husband-wife dyads allows the findings of the present study to be generalizable to a large demographic.

Second, marriage provides a unique context where trust has both individual- and dyadic-level ramifications for couples' subjective well-being. Drawing from \citeauthor{bowenUseFamilyTheory1966}'s \citeyear{bowenUseFamilyTheory1966} family system theory, we view family as a social system and emotional unit, where family members are interdependent and mutually influential. This theory underpins the relational nature of trust within dyadic relationships, where not only is there an intra-couple correlation in trust and well-being, but the trust of one partner may also spill over to influence the other's well-being. Similar dyadic interactions with regard to other spousal characteristics have been observed, such as the negative effects of husband's role strain on their own and their spouse's marital satisfaction \parencite{brockLongitudinalInvestigationStress2008} and the associations between individuals' reemployment and couples' well-being \parencite{scheuringDoesFixedTermEmployment2021}. This study extends this line of literature to examine the individual- and dyadic-level implications of trust for couples' subjective well-being.

Third and lastly, marriage serves as the primary locus exhibiting gendered power dynamics embedded in husband-wife interactions. "A basic fact about family trust," stated by \textcite[p~27]{barberLogicLimitsTrust1983}, "is male-husband-father supremacy, which extends over the wife and the children." This is especially pronounced in the Chinese context, where scholars have observed a resurgence of patriarchal values in recent decades \parencite{jiUnequalCareUnequal2017}. It is, therefore, likely that characteristics of the husband, including levels of trust, predominantly influence couples' subjective well-being, outweighing those of the wife.

Due to the strong relevance of particularized trust to intimate relationships within the trust radius, I anticipate that particularized trust will exert a dyadic-level influence on subjective well-being, whereas generalized trust, directed towards strangers or out-group members, is unlikely to extend its benefits beyond individuals themselves. Based on these considerations, I propose the following hypotheses:

\begin{quote}
    \textbf{H2a.} Particularized trust is positively associated with the spouse's life satisfaction and happiness and negatively associated with the spouse's depression, with husbands exerting a greater influence than wives.

    \textbf{H2b.} Generalized trust is not associated with the spouse's life satisfaction and happiness, and not associated with the spouse's depression.
\end{quote}

\subsubsection{The Relational Pathways as Mediating Mechanisms}

Viewing trust as a relational construct, this study further explores the relational pathways through which trust, especially particularized trust, influences individuals' and their spouse's subjective well-being. Previous studies have provided valuable insights on the individual level. \textcite{luLongitudinalEvidenceSocial2019}, featuring a comprehensive mediation analysis, found that social ties accounted for around 10\% of the total effects of generalized trust on individuals' happiness. Other studies indicated that particularized trust towards family members enhances relational satisfaction and strengthens interpersonal relationships, both of which are crucial for subjective well-being \parencite{adilRoleTrustMarital2013,fitzpatrickAttachmentTrustSatisfaction2017,shekMaritalQualityPsychological1995,wongExaminationRelationshipTrust2002}.

This study is concerned with the relational mechanisms mediating the associations between trust and subjective well-being at both the individual and dyadic levels. Specifically, I identify marital satisfaction and interpersonal relationships as two key mediators that could explain these associations. This leads to the third hypothesis:

\begin{quote}
    \textbf{H3.} Marital satisfaction and interpersonal relationships mediate the associations between trust and couples' subjective well-being.
\end{quote}
