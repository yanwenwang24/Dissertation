\section{Theoretical Framework}
\label{sec:ch4-theoretical-framework}

\subsection{Education and Mate Selection}

Among various factors influencing mate selection, education is one of the most studied, for both theoretical and practical reasons \parencite{blossfeldEducationalAssortativeMarriage2009,kalmijnIntermarriageHomogamyCauses1998}. First, individuals seek in their spouse not only attractive socioeconomic resources but also compatibility in intangible qualities such as beliefs, attitudes, tastes, and knowledge structures; education serves as a reliable proxy for both \parencite{kalmijnIntermarriageHomogamyCauses1998}. As women's economic roles in marriage increasingly resemble those of men, education has become the primary "filter" to reduce the uncertainty of both spouses' changing attributes in the long run \parencite{oppenheimerTheoryMarriageTiming1988,vanbavelReversalGenderGap2018}. Second, educational expansion and the convergence of gender education gaps have led both men and women to spend prolonged periods in school, postponing their entry into first marriages \parencite{jiangMarriageSqueezeNeverMarried2014}. This renders schools, particularly post-secondary institutions—which are homogeneous in age and heterogeneous in gender and thus already an ideal marriage market—increasingly vital venues for partner searching \parencite{blossfeldWhoMarriesWhom2003,jiangMarriageSqueezeNeverMarried2014}. Third, education changes relatively little after first marriage, making it a convenient indicator of marital sorting, especially since other spousal information at the time of union formation is often unavailable in most censuses and national surveys.

China presents a unique context for examining marital sorting by education. Confucian culture highly values education and considers it both as a proxy for and an avenue towards social status \parencite{chenEducationFeverChina2021,guSacrificeIndebtednessIntergenerational2022}. Education is also associated with socioeconomic gains, particularly during periods when the government allocated privileged jobs—mostly in bureaucracies and state-owned units—to university graduates \parencite{bianChineseSocialStratification2002}. Therefore, education is among the most intensely sought-after attributes in mate selection in China.

Individual's choice of "whom to marry" is contingent upon both structural conditions and assortative mating preferences \parencite{kalmijnIntermarriageHomogamyCauses1998,oppenheimerTheoryMarriageTiming1988}. Structural conditions refer to the availability of potential partners within certain educational categories and an appropriate age range, while assortative mating preferences reflect individuals' inclinations towards specific types of educational matches—whether to marry upwards, downwards, or an equally educated spouse. Below, we review the literature on factors influencing marital sorting by education.

\subsection{Assortative Mating Preferences}

Mate selection is a non-random process. Rather, it is shaped by preferences for certain types of partners and structural conditions that enable such preferences. Inclinations towards spouses' (dis)similarity to oneself (i.e., assortative mating preferences) have undergone significant changes over time.

Homogamy is the dominant type of educational match in almost all countries \parencite{dehauwReversedGenderGap2017,esteveEndHypergamyGlobal2016,mareFiveDecadesEducational1991,schwartzTrendsVariationAssortative2013}. This is largely due to individuals' emphasis on spousal compatibility in intangible qualities and the competition for attractive socioeconomic resources associated with higher education \parencite{kalmijnStatusHomogamyUnited1991,kalmijnEducationalGradientMarriage2013,mareFiveDecadesEducational1991,smitsEducationalHomogamy651998}. These preferences often lead to highly stratified patterns of homogamy, where the most attractive candidates pair with each other, leaving those less sought after with fewer options.

In many transitioning societies, including China, scholars have observed rising homogamy \parencite{domanskiEducationalHomogamy222007,dongTrendsEducationalAssortative2023,hanTrendsEducationalAssortative2010,katrnakChangePrevalencePreference2020,mullerEducationYouthIntegration2005}. In China, changes in homogamy coincided with the end of several tumultuous historical events that once undermined the value of education and disrupted mate selection—such as the Send-Down Movement (mid-1960s to 1978) and the Cultural Revolution (1966-1976), and the onset of Economic Reforms, during which higher education was revitalized and became increasingly salient in determining income and occupational status for both genders \parencite{hanTrendsEducationalAssortative2010}. Consequently, educational homogamy faced fewer disruptions and intensified over time.

Gender role norms play an important role in shaping assortative mating preferences. While hypergamy aligns with traditional gender roles that assign men to superior positions and women to subordinate ones, hypogamy often represents a threat to male dominance and is negatively associated with both partners' subjective well-being \parencite{potarcaAreWomenHypogamous2022}. In many Western and developed countries, traditional gender values have weakened due to progress towards gender equality \parencite{bertrandSocialNormsLabour2021,kalmijnEducationalGradientMarriage2013}. \textcite{leeschStructuralOpportunitiesAssortative2024} suggested that, after accounting for the educational composition, assortative mating preferences have had only an insignificant, or at best, weak effect on the dynamics between hypergamy and hypogamy in Europe.

Nevertheless, this may not be the case in China. On the one hand, women have made substantial advancements in educational attainment and, with it, a stronger endorsement of gender egalitarianism \parencite{yeungHigherEducationExpansion2013}. On the other hand, there has been a notable resurgence of patriarchal values following the economic reforms \parencite{jiUnequalCareUnequal2017,muMaritalAgeHomogamy2014}. Women's labor participation rates, once among the highest globally, have declined, and their burdens of unpaid labor relative to men's have increased across cohorts \parencite{luoGenderDivisionHousehold2018}. Therefore, there are likely both continuities of and departures from the conventional preference for hypergamy, which may help explain the co-presence of the persistently low levels of hypogamy and the converging gender gaps in education.

\subsection{Structural Conditions: Educational Expansion and Gradients in Marriage Rates}

Apart from assortative mating preferences, educational sorting outcomes in marriage are also shaped by the structural conditions of the marriage market, specifically the educational compositions of both male and female marital candidates \parencite{kalmijnIntermarriageHomogamyCauses1998,oppenheimerTheoryMarriageTiming1988}. Structurally speaking, individuals are more likely to meet and marry members of an educational group that is larger in relative size \parencite{blauInequalityHeterogeneityPimitive1977,blauHeterogeneityIntermarriage1982}.

Over recent decades, educational expansion has brought about dramatic change to the educational structure in China \parencite{yeungHigherEducationExpansion2013}. In 1949, when the People's Republic of China was founded, about 80\% of the population was illiterate or semi-illiterate, with an even higher percentage among women. By 1964, following campaigns to eradicate illiteracy and the expansion of primary and middle schools, this number had been halved \parencite{treimanTrendsEducationalAttainment2013}. Although the following decade saw higher education thrown into turmoil by the Cultural Revolution, educational expansion did not halt. Though the quality of education was substandard, a short-lived but large-scale expansion of rural secondary schools led to a surge in high school graduates \parencite{sicularBigExpansionRural2022}. The 1980s and beyond saw a revival of higher education, culminating in the state-implemented college expansion in 1999 \parencite{treimanTrendsEducationalAttainment2013}.  To expand college enrollment, public education expenditures—especially in post-secondary education—were increased, colleges were extended by building multiple campuses, and a private college system was instituted \parencite{yeungHigherEducationExpansion2013}. These changes led to substantial growth in the highly educated population and further convergence of gender educational gaps \parencite{treimanTrendsEducationalAttainment2013}.

From a structural perspective, educational expansion is expected to exert dual impacts on marital sorting outcomes. On the one hand, the increasing heterogeneity of educational compositions—from an illiterate majority to an educationally diversified population—may lead to a decline in homogamy. On the other hand, the growing similarity in education between genders is likely to promote homogamy and even hypogamy.

However, not all individuals eventually marry, and marriage rates differ across education levels and genders. This may result in a divergence between the educational composition of the general population and that of the married population. In China, highly educated women are more likely to delay or forgo marriage than their less-educated counterparts \parencite{huGenderEducationExpansion2023,jiTraditionModernityLeftover2015}. A whole package of family obligations and expectations associated with marriage—such as the quick transition to parenthood \parencite{tongResistantChangeTransition2017}, the double burdens of paid and unpaid labor \parencite{luoGenderDivisionHousehold2018}, and the demands of intensive mothering \parencite{guWhyChineseAdolescent2021,guSacrificeIndebtednessIntergenerational2022}—has made marriage particularly unappealing for highly educated women. Additionally, men with lower education are often squeezed out of the marriage market and display lower marriage rates than men with higher education \parencite{jiangMarriageSqueezeNeverMarried2014}. The lower marriage rates among highly educated women and less-educated men may lead to structural changes in the marriage market that favor hypergamy while discouraging hypogamy.

Thus, it is important to account for both assortative mating preferences and structural conditions, within and beyond the married population, to fully understand the changing patterns of marital sorting and contextualize these changes within the broader sociodemographic landscape. However, the conventional research approach, which often solely focuses on assortative mating preferences while holding constant the educational distributions of married men and women, may fall short for such goals. As we will explain later, a decomposition approach addresses these gaps and helps to disentangle the respective contributions of assortative mating preferences and structural changes.

\subsection{The Urban-Rural Divide in China}

The urban-rural divide in China is so vast that the regions seem like two different worlds \parencite{rozelleInvisibleChinaHow2020,sicularUrbanRuralIncomeGap2007,treimanDifferenceHeavenEarth2012}. Rapid urbanization over the past few decades, which transformed China from a rural-majority to an urban-majority country, has not bridged the gap between these two worlds. Rural regions, which lag behind in nearly every aspect of social life, also differ from urban areas in marital sorting outcomes by education \parencite{duTrendsEducationalAssortative2023}.

One of the main causes of this divide is the uneven, highly stratified educational expansion \parencite{yeungHigherEducationExpansion2013}. The initial phase of the educational expansion benefited rural areas more due to its focus on illiteracy eradication and the expansion of primary and middle schools \parencite{hannumPoliticalChangeUrbanRural1999,sicularBigExpansionRural2022}. Despite so, urban-rural educational disparities persisted. Higher education remained concentrated in cities and was virtually non-existent in rural regions. This situation was not alleviated by the ensuing economic reforms, which funneled educational resources predominantly to colleges and universities in cities and accelerated the flow of human capital from rural to urban areas, with higher education being one of the few means to upgrade one's rural status to an urban one \parencite{duTrendsEducationalAssortative2023,yeungHigherEducationExpansion2013}. Moreover, while gender gaps in education have been converging in both urban and rural areas, the pace of change has been much faster in urban settings.

Marriage, as an institution, is more strongly entrenched among the rural population, where marriage rates are higher, and divorce is less common than in urban areas \parencite{luoChineseTrendsAdolescent2020}. This disparity may be attributed to differences in culture and structural conditions. Given the higher proportion of highly educated population in urban areas, especially highly educated women, not only attitudes and ideologies towards marriage may be more diverse and cosmopolitan than in rural areas, the imbalanced "supply" of marriage candidates may also contribute to gendered tendencies towards marriage \parencite{huGenderEducationExpansion2023}.

Educational assortative mating preferences also differ between urban and rural regions. \textcite{duTrendsEducationalAssortative2023} examined trends in the correlation between spouses' relative educational standings and found that urban couples' educational statuses were consistently more strongly correlated than those of rural couples. This was largely due to the stronger prevalence of homogamy among highly educated couples, who were more numerous in urban areas. Additionally, previous studies have found that, due to the persistence of patriarchal values and the greater need to use marriage as a channel for upward mobility, rural women exhibited stronger preferences for hypergamy than their urban counterparts \parencite{weiUnderstandingHypergamousMarriages2016}.

In this study, we aim to explain the variations in marital sorting by education not only across cohorts but also across urban and rural China, focusing on the respective contributions by educational expansion, gender-specific marriage rates across the educational spectrum, and assortative mating preferences.
