\section{Discussion}
\label{sec:ch4-discussion}

Viewing both trust and well-being as multidimensional and relational constructs, this study explores the influence of generalized and particularized trust, both measured using survey items sensitive to the trust radius in China, on couples' subjective well-being across dimensions of life satisfaction, happiness, and depressive symptoms. Using dyadic, longitudinal data from the China Family Panel Studies, I found that particularized trust significantly improved individuals' subjective well-being across all three dimensions, whereas generalized trust had only a marginal impact on reducing men's depression. Furthermore, the benefits of particularized trust spilled over from husbands to wives but not vice versa. Both the within- and between-individual effects of particularized trust were channeled through relational pathways: improved marital satisfaction and interpersonal relationships.

With these findings, the study identified the source of well-being benefits of trust for married couples in China: particularized trust toward people within close relations, as opposed to generalized trust toward strangers. In fact, that the benefit of particularized trust outweighs generalized trust is not an unprecedented finding in East Asia. \textcite{sungIngroupTrustSelfrated2020} found that in-group trust towards closed ones was strongly associated with self-rated health in China, Japan, Korea, and Taiwan, whereas trust towards out-group members positively influenced health only in Taiwan. Similarly, \textcite{sungTrustHealthCrossnational2019}, using data from the WVS, noted that the positive influence of particularized trust on health was almost universal, in contrast to the varied directions and magnitudes of the associations between trust toward strangers and health across countries. By extending this line of research to include subjective well-being, the present study challenges the conventional view that generalized trust invariably contributes positively to subjective well-being and underscores the importance of differentiating various types of trust in specific contexts \parencite{delheyHowGeneralTrust2011,nannestadWhatHaveWe2008}. In contexts such as China, where the trust radius is narrow, and social interactions emphasize close relationships, particularized trust is more relevant and influential for individuals' subjective well-being than generalized trust.

Findings of this study further provides insights into the relational nature of trust. Within marital relationships, particularized trust spilled over its benefits from husbands to wives, but not in the reversed way. These gendered dynamics signifies the male-husband-father supremacy, where husbands often have higher earnings, more advanced education, and due to patriarchal norms, greater power in decision-making and overall control of family life \parencite{jiUnequalCareUnequal2017}. Literature from other fields has reported gendered patterns within dyadic interactions. For instance, husbands' class positions and economic interests had a strong impact on their wives' voting behaviors, while their own behaviors were barely affected by their wives \parencite{dirkdegraafHusbandsWivesVoting1992}. In China, \textcite{wangDoesHappinessContemporary2019} found that the husband's party membership had spillover effects on the wife's happiness, but not the other way around. These studies collectively suggest that husband's characteristics, including trust, have a more significant influence on wives than vice versa.

The relational nature of trust was further supported by mediation analyses. Of notable significance are the moderate to strong mediating effects of marital satisfaction and interpersonal relationships, as they explain both the direct and spillover effects of particularized trust on couples' subjective well-being. These findings position particularized trust as a crucial ingredient for marital harmony and high-quality interpersonal relationships, both of which are essential to individuals' and their spouse's subjective well-being \parencite{fitzpatrickAttachmentTrustSatisfaction2017,rusbultCommitmentTrustClose1999}.

Despite the strengths, this study is not without limitations. First, due to data limitations of the CFPS, several potential relational mechanisms, such as perceived emotional support—an important outcome of trust and predictor of well-being—were not examined. Second, the study spanned a four-year period and did not capture how the well-being implications of trust might evolve across various stages of marriage. Third, because of the limited sample size, the study was unable to assess whether the gendered patterns of spillover effects might vary according to couples' socioeconomic status, power dynamics, and gender role attitudes. Fourth, I restricted samples to couples who remained continuously married between 2014 and 2018, leaving out those whose marriages ended in divorce or separation. This exclusion could lead to selection bias and prevented us from exploring the associations between low levels of trust and marriage dissolution.

These findings and limitations of the present study open up new avenues for the scholarship on trust and well-being. Future research may explore the within- and between-individual effects of trust across different types of dyads and within broader social networks, continuing to bridge the gaps between individual- and contextual-level studies. It may also examine trust dynamics across social strata and multiple stages of relationships, and most importantly, conduct comparative studies on cross-national differences in the well-being implications of various types of trust. These research directions would deepen our understanding of how and why trust is an essential ingredient for individuals' and their related ones' well-being.
