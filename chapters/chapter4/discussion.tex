\section{Discussion}
\label{sec:ch4-discussion}

Using data from China's Census, our study examined trends in marital sorting outcomes by education across consecutive cohorts born between 1946 and 1985. We first described how the prevalence of homogamy, hypergamy, and hypogamy varied across cohorts and urban/rural regions, and how the three driving forces (i.e., educational expansion, gender- and education-specific marriage rates, assortative mating preferences) shifted correspondingly. We then employed a decomposition approach to disentangle the respective contributions of these three factors to the observed educational sorting outcomes in marriage.

Our descriptive findings align with previous studies in China \parencite{dongTrendsEducationalAssortative2023,hanTrendsEducationalAssortative2010,shiSevenDecadesEducational2019}. Specifically, we observed a U-shaped curve for homogamy, an inverted U-shaped curve for hypergamy, and an initial rise in hypogamy followed by stagnation. Throughout the course of educational expansion, the educational composition within both the married population and the general population has become more heterogeneous within each gender but more homogeneous between genders \parencite{yeungHigherEducationExpansion2013}. Changes in educational structure could have led to a substantial increase in hypogamy, as seen in other contexts, if not for the rapidly declining marriage rates among highly educated women and the intensified assortative mating preferences against heterogamy, especially hypogamy \parencite{hanTrendsEducationalAssortative2010,huGenderEducationExpansion2023}. These unique dynamics highlight the intricate interplay between tradition and modernity in China, reflected in the co-existence of both rejection and adherence to traditional marriage norms: the higher a woman's educational attainment, the less likely she is to marry; but if she does, the less willing she is to marry downwards.

Applying the decomposition approach proposed by \textcite{leeschDecomposingTrendsEducational2023}, we quantitatively explained why these trends occurred in China, a transitional country with rapid but uneven development over the past century. Our decomposition results show that the initial decline in homogamy and the corresponding increase in heterogamy across cohorts born before 1965 were driven entirely by educational expansion, while the gender-education-specific marriage rates and assortative mating preferences played negligible roles. For later birth cohorts, sustained educational expansion—characterized by a further convergence of gender education gaps—promoted homogamy and hypogamy while discouraging hypergamy, outweighing the opposing force of a steeper decline in marriage rates among highly educated women. These cohorts also exhibited intensified preferences for homogamy and against heterogamy. These three factors jointly explained the patterns of increasing homogamy, declining hypergamy, and stagnant hypogamy among later birth cohorts.

These results differ from those found by \textcite{leeschDecomposingTrendsEducational2023} in Ireland, where, in addition to educational expansion, the increasing marriage rates among highly educated women contributed to rising homogamy and declining hypergamy, whereas the impact of assortative mating preferences were minimal. These differences suggest that the unique demographic transitions and the unparalleled importance of education in shaping China's sociodemographic changes contributed to the complex dynamics among structural conditions, marriage rates, and assortative mating preferences.

Unexpectedly, the urban-rural differences in marital sorting outcomes by education have been narrowing across cohorts born between 1966 and 1985. Furthermore, upon decomposing the differences across urban and rural regions, we found that a steeper decline in marriage rates among highly educated women in rural regions was a significant force narrowing the urban-rural gap in homogamy. Regarding heterogamy, diminishing differences in the impact of educational expansion were the main contributor to this convergence. Among younger cohorts born between 1981 and 1985, the rates of hypergamy converged across urban and rural areas, and the remaining differences in hypogamy were largely due to educational structures and stronger assortative mating preferences against hypogamy in rural areas. Those findings suggest that, despite the persistent socioeconomic disparities, marriage was one of the primary sites where the integration of urban and rural China took place.

This study, however, is not without limitations. First, the decomposition approach, in its strict form, assumes that educational expansion, education-specific marriage rates, and assortative mating preferences are independent forces \parencite{leeschDecomposingTrendsEducational2023}. This assumption may not hold in reality, as the expansion of higher education could marginalize certain groups, particularly less-educated men, further squeezing them out of the marriage market and, in turn, affecting the educational gradients in marriage rates \parencite{jiangMarriageSqueezeNeverMarried2014}. Similarly, assortative mating preferences may shift due to the increasing prominence of education during periods of educational expansion. Nevertheless, this limitation is common in other analyses using the decomposition or log-linear approach and does not weaken the strength of our study in distinguishing the respective contributions of the three aforementioned factors.

Second, our study is limited by the exclusion of marriages between spouses with different \textit{hukou} statuses. We excluded this group for analytical reasons. However, this is a very selective group, representing less than 5\% of all marriages across cohorts in our sample \parencite{duTrendsEducationalAssortative2023}. Moreover, as mentioned earlier, \textit{hukou} conversion often requires stringent conditions. Therefore, we do not think this decision will affect the results significantly. Still, we want to note that our findings on the diminishing urban-rural gaps in marital sorting by education are limited to spouses who share the same \textit{hukou} status.

Third, the four-level education may obscure stratification mechanisms within each level \parencite{fengRevisitingHorizontalStratification2022}. At the lower end of the spectrum, illiteracy and primary school education were grouped together, meaning marriages between individuals with these two statuses were classified as homogamous. Likewise, vocational colleges, which have become more prevalent among younger cohorts, are often considered inferior to four-year colleges due to their less stringent entry requirements and lower socioeconomic returns. To ensure a sizable population within each level across cohorts, we were unable to utilize finer educational gradients. Furthermore, as aforementioned, this categorization is still effective in capturing varied socioeconomic returns in contemporary China.

Lastly, the census data did not include information on cohabitation, which limited our ability to account for educational sorting outcomes in this relationship. However, this is unlikely to significantly affect our results, as cohabitation remains uncommon and is largely viewed as a predecessor to marriage in China \parencite{muChangingPatternsDeterminants2022}.

These limitations do not diminish the strength of our study. To the best of our knowledge, this is the first study to systematically account for both preferences and structural conditions in understanding variations in educational sorting outcomes in marriage across cohorts and between urban and rural areas in China. Additionally, we captured structural influences within both the married and the general population, providing novel evidence on how marriage plays a role in the nuanced implications of modernization. Despite various changes, the dominance of homogamy may further intensify social stratification. While some highly educated women depart from traditional gendered marital norms by opting out of marriage, those who do marry tend to adhere more strongly to hypergamy than their less educated counterparts. Surprisingly, despite the widened urban-rural socioeconomic disparities, patterns of marital sorting in urban and rural China are converging, highlighting the pervasiveness of changing family ideals and forms. We encourage further research to extend this line of inquiry to other contexts and through a cross-national perspective.
