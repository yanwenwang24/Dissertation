\section{Introduction}
\label{sec:ch4-introduction}

Marital sorting by education has far-reaching implications for social and gender inequality in modern societies \parencite{blossfeldEducationalAssortativeMarriage2009,blossfeldWhoMarriesWhom2003,schwartzTrendsVariationAssortative2013}. Marriages between partners with equal educational levels, known as educational homogamy, may impede mobility across educational groups and reinforce social stratification within and across generations \parencite{kalmijnStatusHomogamyUnited1991,kalmijnIntermarriageHomogamyCauses1998,mareFiveDecadesEducational1991,smitsEducationalHomogamy651998}. On the other hand, educational heterogamy—further specified as educational hypergamy if women marry up and hypogamy if women marry down—is perceived as an important indicator of social openness and profoundly influences the conjugal power dynamics, as reflected in the gendered division of labor within households and gender role attitudes in broader society \parencite{bakerMarriageSpecializationGender2007,dribeEducationalHomogamyGenderSpecific2013,eeckhautEducationalHeterogamyDivision2014,eikaEducationalAssortativeMating2019,vanbavelEducationalPairingsMotherhood2017}. Given these implications, there has been a rich and growing literature examining the changing patterns of educational sorting outcomes in marriage and the underlying factors driving these changes \parencite{eratEducationalAssortativeMating2021,esteveGenderGapReversalEducation2012,esteveEndHypergamyGlobal2016,hirschlEightDecadesEducational2024,leeschDecomposingTrendsEducational2023,leeschStructuralOpportunitiesAssortative2024}.

In the extant literature, trends such as the persistent dominance of homogamy, declining hypergamy, and rising hypogamy often reflect structural changes in the marriage market—most notably, the reversal of gender gaps in education favoring women amidst educational expansion \parencite{dehauwReversedGenderGap2017,eratEducationalAssortativeMating2021,esteveGenderGapReversalEducation2012,esteveEndHypergamyGlobal2016,schwartzTrendsEducationalAssortative2005}. However, China presents an idiosyncratic case. Over the past few decades, the prevalence of homogamy in China has followed a U-shaped curve—initially high among earlier cohorts, declining among those impacted by the Cultural Revolution, and rising again with the transition to a market economy \parencite{dongTrendsEducationalAssortative2023,hanTrendsEducationalAssortative2010,shiSevenDecadesEducational2019}. Among the 1995-2001 cohorts, over 65\% of all marriages were homogamous \parencite{hanTrendsEducationalAssortative2010}, one of the highest rates globally \parencite{dehauwReversedGenderGap2017,eratEducationalAssortativeMating2021,katrnakChangePrevalencePreference2020}. Moreover, despite the decline in hypergamy, hypogamy has remained remarkably uncommon, fluctuating around 10\% of all marriages across cohorts \parencite{hanTrendsEducationalAssortative2010}. This is at odds with the convergence of gender education gaps during periods of mass educational expansion, implying a stark discrepancy from patterns predicted by structural conditions of the marriage market—a phenomenon rarely observed elsewhere.

Against this backdrop, this study aims to understand not only how but also why these changes occurred as they did in China. Previous studies primarily focused on the role of assortative mating preferences by holding constant the educational composition of married couples \parencite{dongTrendsEducationalAssortative2023,hanTrendsEducationalAssortative2010,huEducationalAgeAssortative2019,qianAssortativeMatingEducation2017}. These studies found a significant and steady increase in assortative mating preferences and the strength of couples' educational associations over the past century, interrupted only by the Cultural Revolution \parencite{dongTrendsEducationalAssortative2023,hanTrendsEducationalAssortative2010}. However, several research gaps remain unaddressed. First, despite profound structural changes in the marriage market, it remains unclear how these changes have influenced the relative prevalence of homogamy, hypergamy, and hypogamy, independently and jointly with changes in assortative mating preferences. Second, focusing exclusively on married couples does not adequately capture the broader sociodemographic transitions in China. Notably, educational expansion has led to converging educational attainment yet diverging educational gradients in marriage rates across genders \parencite{gruijtersTrendsEducationalMobility2019,yeungHigherEducationExpansion2013}. Marriage has been declining at varying rates across genders and educational levels, with the sharpest declines observed among highly educated women \parencite{huGenderEducationExpansion2023,raymoMarriageFamilyEast2015}. Those sociodemographic shifts are prominent in shaping structural opportunities for mate selection, which have not been sufficiently examined. Third, despite the massive scale, educational expansion has been uneven across the urban and rural areas, which escalated disparities in the marriage market across urban and rural areas \parencite{sicularBigExpansionRural2022,yeungHigherEducationExpansion2013}. Yet, there has been limited research systematically examining marital sorting by education with attention to the growing urban-rural disparities. Addressing these gaps will provide a more comprehensive, contextualized understanding of the forces shaping "who marries whom" in China.

Our study aims to fill those gaps by answering the following research questions: How have patterns of marital sorting by education evolved across cohorts in China? How do these patterns differ between urban and rural areas? To what extent can these patterns be explained by changes in assortative mating preferences and structural conditions within and beyond the marriage market? To answer these questions, we selected respondents from consecutive cohorts born between 1946 and 1985, using data from the 1982, 1990, 2000, and 2010 Chinese Censuses. Additionally, we included a subsample of respondents born between 1966 and 1985 for a stratified analysis examining differences between urban and rural areas. We employed the decomposition approach proposed by \textcite{leeschDecomposingTrendsEducational2023} to unpack the specific contributions of three key forces: educational expansion, education-specific marriage rates, and assortative mating preferences.

As far as we are aware, this study is among the first to extend the works of \textcite{leeschDecomposingTrendsEducational2023,leeschStructuralOpportunitiesAssortative2024} beyond the European context by systematically explaining trends in marital sorting by education in China—a country that has undergone rapid development and sociodemographic transitions over the past century. Our findings provide a more nuanced understanding of the intricate, still-unfolding interplay between subjective preferences and structural conditions during China's modernization. These dynamics predominantly favor homogamy, are somewhat divided on hypergamy, and are counteractive with respect to hypogamy; to some extent, they also contribute to the narrowing of the urban-rural divide.

The remainder of this article is organized as follows: We begin by reviewing the theoretical framework for marital sorting by education, followed by an introduction to the Chinese context, with particular emphasis on the urban-rural divide. Next, we provide details on the data, measures, and analytical strategies. We then describe the observed trends and present our empirical findings. Finally, we conclude by summarizing the results, discussing their limitations, and outlining implications for future research.
