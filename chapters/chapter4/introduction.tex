\section{Introduction}
\label{sec:ch4-introduction}

Trust, "one of the most important synthetic forces within society," is vital for establishing and maintaining meaningful relationships \parencite[p.~326]{simmelSociologyGeorgSimmel1964}. Trust enables us to develop secure attachments with those close to us and to exchange information and cooperate with those more distant \parencite{rusbultCommitmentTrustClose1999,schilkeTrustSocialRelations2021,simpsonFoundationsInterpersonalTrust2007}. Within the larger society, trust, a core component of social capital, fosters inclusiveness, civic engagement, and collective actions, all of which are foundational for economic, democratic, and cultural prosperity \parencite{fukuyamaTrustSocialVirtues1996,fukuyamaSocialCapitalCivil2001,putnamBowlingAloneCollapse2001}.

Since \citeauthor{easterlinDoesEconomicGrowth1974}'s (\citeyear{easterlinDoesEconomicGrowth1974}) pioneering work, individuals' subjective well-being, the subjective experience and evaluation of various domains of life, has garnered considerable research attention. Trust plays a prominent role in this context, at times even surpassing the importance of income \parencite{bjornskovHappyFewCross2003,helliwellHowsLifeCombining2003}. Numerous studies in both developed and developing societies have demonstrated that individuals with a higher level of trust are happier \parencite{luLongitudinalEvidenceSocial2019,rodriguez-poseSocialCapitalIndividual2013}, more satisfied with life \parencite{zhangSocialTrustSatisfaction2020}, and exhibit fewer depressive symptoms \parencite{fahmiDoesYourNeighborhood2019,fujiwaraProspectiveStudyIndividuallevel2008}. The benefits of trust extend to collectives as well; those residing in communities or nations with higher aggregated levels of trust have better overall well-being, controlling for their own level of trust \parencite{helliwellHowsLifeCombining2003,tovWellBeingNationsLinking2009}.

Nevertheless, there has been a lack of consensus on what trust, an inherently multidimensional and multifaceted construct, designates across contexts \parencite{nannestadWhatHaveWe2008}. Previous studies on social capital have often used generalized trust to represent a moralized belief in human goodness \parencite{adedejiExaminingPathwaysGeneral2023,helliwellHowsLifeCombining2003,helliwellSocialContextWell2004}. However, given strong familism and limited civic participation in China, scholars have questioned whether generalized trust is as significant for subjective well-being in China as it is in Western societies, and whether it can be accurately measured by the standard survey item: “Generally speaking, would you say that most people can be trusted or that you need to be very careful in dealing with people” \parencite{delheyHowGeneralTrust2011,liParticularizedTrustGeneralized2002}. In societies where the trust radius—the maximum distance between the subject and objects of trust—is narrow, respondents, when asked about "most people," may draw on relationships within their immediate circles of contacts rather than the general population \parencite{liParticularizedTrustGeneralized2002}. Consequently, responses may misidentify particularized trust in known, familiar individuals as the intended generalized trust, obscuring scholars from estimating how various types of trust affect subjective well-being.

Therefore, the first contribution of this study is to contextualize trust in the Chinese context by distinguishing between generalized and particularized trust and investigating their well-being implications separately. For more precise measures, I employ survey items that clearly specify the objects of trust within and beyond the trust radius for measuring particularized and generalized trust, respectively.

Trust is not only a multidimensional construct but also a relational one \parencite{lewisTrustSocialReality1985}. However, previous literature, primarily focusing on individuals or collectives as units of analysis, have offered limited evidence on how one person's trust may extend its benefits to others, especially their close ones, via relational pathways. Viewing trust as a relational construct within family systems \parencite{bowenUseFamilyTheory1966}, this study explores the dyadic spillover effects of trust in marital relationships, where husbands and wives are interdependent and mutually influential, and trust, especially particularized trust, may extend its benefits beyond the individual to influence their spouse's subjective well-being. The focus on married couples is twofold: not only does the prevalence and centrality of marriage in China make this study relevant to a large demographic \parencite{jiangMarriageSqueezeNeverMarried2014,yanChineseFamiliesUpside2021}, but marital relationships also serve as a primary locus for observing the gendered power dynamics embedded in husband-wife interactions \parencite{barberLogicLimitsTrust1983,jiUnequalCareUnequal2017}. Given the persistent male-husband-father supremacy, I anticipate that husbands' trust will have a greater impact on couple' subjective well-being than wives' trust.

This study further contributes to the literature by examining the relational pathways through which trust influences individuals' and their spouse's well-being. Previous research on intimate relationships has found that trust improves relational satisfaction and broadens interpersonal relationships, which are integral to subjective well-being \parencite{adilRoleTrustMarital2013,fitzpatrickAttachmentTrustSatisfaction2017,luLongitudinalEvidenceSocial2019,shekMaritalQualityPsychological1995,wongExaminationRelationshipTrust2002}. Building on this line of literature, I explore the relational pathways—marital satisfaction and interpersonal relationships—as mediating mechanisms that could explain the well-being implications of trust on both the individual and dyadic levels.

To this end, this study utilizes a lagged Actor-Partner Interdependence Model with Mediation (APIMeM) to assess how generalized and particularized trust influences married couples' subjective well-being across multiple domains, including life satisfaction, happiness, and depression, with attention to gender variations and the relational pathways mediating these associations. The lagged APIMeM is a dyadic and longitudinal approach that isolates and estimates the within- and between-individual influence of trust, while controlling for intra-couple covariances in trust and subjective well-being. It also reduces biases caused by omitted variables by holding constant the baseline well-being outcomes. I selected a nationally representative sample of 6,694 pairs of heterosexual married couples from the China Family Panel Studies in 2014 and 2018.

The article is organized as follows. I start with the conceptualization and measures of trust in the Chinese context, followed by a brief review of empirical evidence on the well-being implications of trust. Next, I contextualize trust within marital relationships, with a focus on gender dynamics and relational mechanisms. Subsequently, I present the data, methods, and findings. I conclude by summarizing findings and discussing limitations and future implications.
