\section{Results}
\label{sec:ch4-results}

\subsection{Descriptive Statistics}

Table~\ref{tab:trust-descriptive-stats} summarizes the descriptive statistics of the sample. Among the three dimensions of subjective well-being, gender differences were found only in depression levels: husbands were less depressed than wives, as evidenced by their lower CES-D20 scores. In terms of trust, particularized trust was significantly higher than generalized trust for both genders. Husbands reported higher levels of particularized trust (M=8.13, SD=1.40) and generalized trust (M=2.08, SD=2.09) compared to their wives, aligning with findings from previous literature \parencite{delheyHowGeneralTrust2011,sungIngroupTrustSelfrated2020}. Additionally, husbands reported higher marital satisfaction ($M$=4.68, $SD$=0.66) but lower scores in interpersonal relationships ($M$=7.13, $SD$=1.92) than their wives.

Regarding sociodemographic characteristics, husbands were on average 49.24 years old ($SD$=12.87), 1.86 years older than their wives. They had more years of education ($M$=8.03, $SD$=4.05) compared to their wives ($M$=6.35, $SD$=4.68). On average, couples had 1.38 children and a per capita household income of 13,035 yuan. The average duration of marriage was 25.3 years, with a standard deviation of 13.04 years.

\subsection{Trust and Couples' Subjective Well-Being}

Table~\ref{tab:apim-results} summarizes the results of the APIMs on couples' subjective well-being. The unconstrained model, which allowed for gender variations, had excellent fits (CFI=1.00, RMSEA=.00, SRMR=.00) and was preferred over the constrained models with poorer fits. This confirms the presence of gender differences in the associations between trust and couples' subjective well-being.

I first looked into the actor effects, specifically how trust influences individuals' own subjective well-being. Results indicated that particularized trust was significantly associated with subjective well-being for both genders. For men, particularized trust measured in 2014 was significantly related to all three dimensions of subjective well-being measured in 2018. After standardization, a one-unit increase in particularized trust corresponded to a 0.06-point increase in life satisfaction and happiness and a 0.11-point decrease in depression, all significant with p-values below 0.001. For women, particularized trust was significantly associated with life satisfaction (0.05) and depression (-0.07) but not with happiness. No gender differences were found except in the aspect of happiness. Therefore, hypothesis H1b, which posits that particularized trust enhances individuals' subjective well-being, is validated.

Next, I examined the partner effects. The results revealed asymmetric associations between individuals' particularized trust and spousal well-being. Partner effects from husbands to wives were significant: a one-unit increase in the husband's particularized trust was associated with a 0.03-point increase in life satisfaction, a 0.04-point increase in happiness, and a 0.04-point decrease in depression. Conversely, partner effects from wives to husbands were only significant for depression (-0.03) but had little impact on life satisfaction and happiness. These gendered patterns suggest that the spillover effects of particularized trust through marital ties on spousal subjective well-being, particularly regarding life satisfaction and happiness, were primarily unidirectional. Husbands had a greater influence on their spouse's well-being compared to wives. These findings validated hypothesis H2, suggesting that particularized trust extends its benefits to influence spousal subjective wellbeing, with husbands being more influential.

As for generalized trust towards strangers, most of its effects on couples' subjective well-being were insignificant. Across three dimensions of subjective well-being, only men's depression (with a decrease of 0.04) showed a significant association with generalized trust. Moreover, the partner effect was insignificant, suggesting that generalized trust did not spill over to affect the spouse's subjective well-being, whether from the husband or the wife. Hypothesis H1a is, therefore, rejected.

\subsection{The Relational Pathways: Marital Satisfaction and Interpersonal Relationships}

I further explored the relational pathways through which trust influences couples' subjective well-being. Table~\ref{tab:apimem-ms-results} and Table~\ref{tab:apimem-ir-results} summarize the results of the APIMeMs, with marital satisfaction and interpersonal relationships included as potential mediators. For model simplicity and improved fit measures, the APIMeMs constrained the paths and did not allow for gender variations \parencite{ledermannAssessingMediationDyadic2011}.

As shown by the results, both marital satisfaction and interpersonal relationships had moderate to strong mediating effects. Specifically, marital satisfaction explained a significant proportion of the effects of particularized trust, accounting for 32.6\%, 27.7\%, and 21.3\% of the actor effects, and 45.5\%, 24.4\%, and 40.9\% of the partner effects on life satisfaction, happiness, and depression, respectively. Interpersonal relationships accounted for 37.3\%, 93.5\%, and 18.4\% of the actor effects, and 49.3\%, 58.3\%, and 29.9\% of the partner effects for the three indicators of subjective well-being, respectively. Notably, both the actor and partner effects of particularized trust on happiness became insignificant once mediated, suggesting that the ability to maintain good interpersonal relationships with others was an especially important pathway through which particularized trust enhances couples' happiness. Therefore, the mediation analyses validated hypothesis H3, suggesting that the positive impact of particularized trust on subjective well-being could be largely explained by improvements in marital satisfaction and interpersonal relationships.

\subsection{Auxiliary Analyses}

For robustness checks, I conducted several auxiliary analyses. First, considering the potential issue of reverse causality, I fit models predicting trust measured in 2018 by subjective well-being measured in 2014. These models had worse fit measures and larger Bayesian information criterion (BIC), suggesting that trust was more likely to be a determinant of subjective well-being, rather than the reversed direction. Additionally, I checked the causal directions between mediators and subjective well-being. Comparison of fit measures and BIC showed that the mediation directions hold better with marital satisfaction and interpersonal relationships being the mediators. Finally, I applied list-wise deletions of respondents with missing variables and found no significant differences compared to the models using FIML.