\section{Results}
\label{sec:ch4-results}

\subsection{The Changing Landscape of Marital Sorting by Education}

Figure~\ref{fig:trends_edu_sorting_marriage} illustrates the patterns of marital sorting by education across the 1946-1985 cohorts, as well as those stratified by urban/rural status across the 1966-1985 cohorts. The solid line represents the actual observed patterns, while the dashed line denotes structural sorting outcomes—rates of homogamy, hypergamy, and hypogamy that would have occurred if husbands and wives were matched randomly within their educational distributions, without patterned assortative mating preferences.

Consistent with previous studies, the share of homogamy followed a U-shaped curve. In the first stage, it declined from 68.0\% to as low as 52.1\% across the 1946-1950 and 1961-1965 cohorts—cohorts whose young adulthood was heavily influenced by tumultuous historical events such as the Cultural Revolution. In the second stage, with the marketization of China, the prevalence of homogamy rebounded rapidly, exceeding 70\% among cohorts born after 1976. Across the 1966-1985 cohorts, homogamy rates followed similar trajectories across urban and rural areas, with the rates being slightly higher in rural regions. Additionally, we observed a stark gap between observed and structural homogamy across cohorts, which widened significantly for those born after 1966, particularly in urban areas. This widening gap suggests an intensifying preference for homogamy.

The proportion of hypergamous marriages followed an inverted U-curve, rising from 27.0\% among the 1946-1950 cohorts to 35.6\% among the 1961-1965 cohorts, then steadily declining to 18.3\% among those born between 1981 and 1985. Though hypergamy rates were initially higher in rural areas, the urban-rural differences converged among young cohorts. The disparities between the observed and structural trends, larger in urban areas, suggest that hypergamy rates would have been higher if not for assortative mating preferences being increasingly against hypergamy.

Regarding hypogamy, after an initial increase from 4.9\% to 12.3\% among older cohorts born in the 1960s or earlier, its prevalence declined following the 1961-1965 cohorts. Since then, it remained stagnant, fluctuating around 10\% across later cohorts. Hypogamy rates in urban areas were consistently higher than in rural regions, but these differences have been declining, due to the decrease in hypogamy in urban regions. The gap between observed and structural hypogamy, was much more pronounced in urban areas than in rural ones. This suggests that, without the increasing disinclination towards hypogamy, its proportion would have been much higher.

\subsection{Structural Changes: Educational Expansion and Gradients in Marriage Rates}

We then computed the relative proportion of men and women at each educational level across cohorts and urban/rural regions, regardless of marital status. Figure~\ref{fig:edu_comp} illustrates the mass education expansion that led to substantial advancements in educational attainment for both genders, a significant narrowing, or even reversal, of gender educational gaps, and considerable disparities between urban and rural regions. Among those born in 1950 or earlier, the majority—66.5\% of men and 83.7\% of women—were either illiterate or had only primary school education. Approximately a quarter (26.3\%) of men and far fewer women (12.3\%) graduated from middle school. Only 5.9\% of men and 3.5\% of women obtained a high school diploma, and college or higher education was virtually non-existent, with only 1.3\% of men and 0.5\% of women achieving this level.

This situation has changed dramatically since then. The proportion of individuals with primary school education or less has dropped significantly, to less than 10\% for both genders among those born in the 1980s. This was accompanied by a considerable increase in middle school education, which accounted for the largest share—around 50\%—for both genders born in the 1980s. The increase in high school education, reaching slightly under one-fifth of the population born in the 1980s, would have been steady if not for the temporary expansion of rural secondary schools during the Cultural Revolution. Higher education has become increasingly popular: 20.8\% of men and an even higher 24.3\% of women born in the 1980s obtained a college degree or higher, indicating a convergence, and even a reversal, of gender gaps in higher education.

However, the urban-rural divide has been exacerbated by educational expansion. Among individuals born in the 1980s, the proportion of those with only primary school education or less has dropped to around 1.5\% in urban areas. In contrast, despite a decline, it remained at 12.4\% for men and 13.6\% for women in rural areas. The most pronounced urban-rural gap is in higher education: 57.3\% of men and 62.2\% of women in urban regions have obtained a college degree or higher, compared to just 5.7\% of rural men and 7.4\% of rural women.

Figure~\ref{fig:gradient} illustrates changes in gender- and education-specific marriage rates. Among men, marriage rates were lower at the two ends of the educational spectrum—those with primary school education or less and those with college or higher education—and comparatively higher among those with middle or high school education. For women, we found consistently negative educational gradients among women across cohorts, with marriage rates decreasing as educational attainment increased. Changes in marriage rates varied more across the educational spectrum for women born after 1970. The most significant decline was among women with college or higher education, with differences from their lower-educated counterparts widening, particularly for women in rural areas. These dynamics would lead to fewer highly educated women in the marriage market, consequently suppressing the increase in hypogamy.

\subsection{Assortative Mating Preferences}

Next, we employed log-linear models to estimate the odds ratios of homogamy, hypergamy, and hypogamy, controlling for the educational composition of husbands and wives, as well as their interaction terms with cohort dummies. We use estimates from these models to indicate changes in assortative mating preferences.

As shown in Figure~\ref{fig:odds_ratio}, Chinese couples exhibited strong preferences for homogamy and against heterogamy. These preferences intensified among cohorts born in the 1960s or later. More specifically, preferences for homogamy were strongest among women with college or higher education, in both urban and rural areas. Preferences against hypergamy were more pronounced among women with lower education. In rural areas, women with high school education were the only group showing a positive preference for hypergamy, even among the youngest cohorts in our sample. Conversely, hypogamy was particularly unfavorable for women with higher education, with those with college or higher education in rural areas exhibiting the strongest disinclination towards hypogamy.

\subsection{Results from the Decomposition Analysis}

Using the decomposition approach, we estimated the respective contributions of educational expansion, gender-education-specific marriage rates, and assortative mating preferences to differences in marital sorting outcomes by education across cohorts and between urban and rural regions. The results are visualized in Figure~\ref{fig:decompose}.

As shown, the initial decline in homogamy and the corresponding rise in hypergamy and hypogamy among the 1946-1950 and 1961-1965 cohorts were driven almost entirely by educational expansion. While older cohorts were predominantly composed of individuals with primary school education or less, those born in the 1960s had a much more heterogeneous distribution of educational attainment, which aggravated structural difficulties in finding an equally educated partner. Meanwhile, educational expansion also led to a narrowing of gender educational gaps, reflected by an increase in hypogamy from 4.9\% in the 1946-1950 cohort to around 12\% in the 1961-1965 cohorts. Across these cohorts, the other two factors—gender-education-specific marriage rates and assortative mating preferences—changed little and contributed negligibly to the overall patterns.

Cohorts born in the 1960s and afterwards experienced a substantial increase in homogamy, a corresponding decline in hypergamy, and stagnation in hypogamy. As shown in Table 1, the underlying causes of these changes were complex. First, the rising homogamy was largely driven by educational expansion and shifts in assortative mating preferences, both of which increasingly favored homogamy. From cohorts born in the 1960s to those in the 1980s—during which the prevalence of homogamy rose from 52.1\% to 72.4\%—changes in educational composition and assortative mating preferences accounted for 60.8\% and 42.9\% of this increase, respectively, while changes in education-specific marriage rates contributed a negative 3.7\%.

Next, the decline in hypergamy among these cohorts was primarily driven by educational expansion and assortative mating preferences against hypergamy. At the same time, although the changing education-specific marriage rates promoted hypergamy among cohorts born in the 1970s and 1980s, their influence was overshadowed by the other two factors. Specifically, hypergamy rates declined from 35.6\% to 18.3\% across cohorts born between the 1960s and the 1980s. Of this decline, 94.9\% was attributed to educational expansion, 23.1\% to assortative mating preferences against hypergamy, while a negative 18\% was due to fewer men with low education and fewer women with high education entering the marriage market.

Lastly, the proportion of hypogamy declined from a peak of 12.3\% among the 1960s cohorts to 9.3\% among the 1980s cohorts. Although educational expansion led to the convergence of educational gaps between genders, creating structural conditions conducive to hypogamy, this effect was offset by two countervailing forces: declining marriage rates among highly educated women and assortative mating preferences, both of which became increasingly unfavorable to hypogamy over time.

Following the main analyses, we decomposed the differences in marital sorting by education between urban and rural regions, as illustrated in Figure~\ref{fig:decompose}. Homogamy rates were consistently higher in rural regions compared to urban ones. The difference widened from 1.6 percentage points among the 1960s cohorts to 4.9 percentage points among the 1975s cohorts, then diminished to 2.3 percentage points among the 1980s cohorts. In rural China, educational expansion and assortative mating preferences more strongly favored homogamy than in urban areas. However, the steeper decline in marriage rates among highly educated women in rural regions worked against homogamy, and this effect intensified across cohorts, contributing to the narrowing of the urban-rural divide in homogamy among the 1980s cohorts.

The urban-rural differences in the prevalence of hypogamy have diminished over time, from a 5.4 percentage point difference among the 1966-1970 cohorts to a small but still significant 2.6 percentage point difference among the 1981-1985 cohorts. Educational expansion was the main contributor. Despite being uneven across urban and rural regions, it has exerted an increasingly similar influence on hypogamy rates among younger cohorts. Aside from educational composition, the remaining urban-rural divide was primarily due to stronger assortative mating preferences against hypogamy in rural areas compared to urban ones.

Lastly, hypergamy rates, initially higher in rural areas, converged across cohorts. From the 1976-1980 cohorts onward, urban-rural differences in hypergamy rates were no longer significant. Although education-specific marriage rates favored hypergamy more in rural regions, the effect was offset by the other two factors: educational composition and assortative mating preferences, both of which more strongly disfavored hypergamy in rural regions compared to urban ones.

Therefore, we conclude that, despite substantial educational disparities, the urban-rural divide in the patterns of marital sorting by education has been narrowing.
