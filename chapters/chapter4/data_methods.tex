\section{Data and Methods}
\label{sec:ch4-data-methods}

\subsection{Data and Sample}

The study utilized data from the China Family Panel Studies (CFPS), a national biennial survey of Chinese communities, families, and individuals. The CFPS used a multistage probability strategy with implicit stratification to draw samples representative of China’s population \parencite{xieSamplingDesignChina2015}. It interviewed around 15,000 individuals at its baseline in 2010 and has accumulated over 50,000 participants over the years. The CFPS provides repeated measures of couples' trust and subjective well-being, as well as other sociodemographic characteristics, making it an ideal resource for dyadic, longitudinal analyses. This study selected samples from the 2014 and 2018 waves. Earlier waves were not included due to missing data on trust in 2010 and on happiness and depression in 2012.

Stepwise sample restrictions were performed. First, I excluded individuals who were not married in either wave, reducing the sample size from 47,252 to 19,149 individuals by 59.47\%. Second, I selected continuously married heterosexual couples who both responded to the surveys, further reducing the sample size by 18.73\% to 7,781 pairs of couples. Third, following Little's Missing Completely at Random (MCAR) test—which indicated that missing data was randomly distributed, I performed list-wise deletions of respondents with missing values for two key variables: subjective well-being and trust. The final analytic sample comprised 6,964 pairs of married couples.

\subsection{Measures}

\textbf{Subjective Well-Being.} The dependent variable, subjective well-being, was assessed across three dimensions: life satisfaction, happiness, and depression. Multidimensional measures are preferred as they represent separable components that may exhibit diverging patterns \parencite{pavotAssessmentSubjectiveWellBeing2008}. Specifically, life satisfaction was measured on a scale of 1 (\textit{very dissatisfied}) to 5 (\textit{very satisfied}) by the survey item: "Are you satisfied with your life?" Happiness was recorded by responses on a 0 (\textit{the lowest}) to 10 (\textit{the highest}) scale to the question "Are you happy?" Depression was evaluated using the CES-D20 scale, with higher scores indicating more severe depressive symptoms \parencite{radloffCESDScaleSelfReport1977}. This multi-item measure demonstrated good reliability, with Cronbach's alpha of 0.86. The CES-D20 scale also showed strong construct validity. As shown in Table 1, it was strongly and negatively correlated with life satisfaction ($r$ = -0.27) and happiness ($r$ = -0.36).  These correlations support the validity of all three measures as reliable indicators of subjective well-being, consistent with findings from previous studies \parencite{dienerSubjectiveWellBeingScience2009}. In the analyses below, I treated the latter measures of subjective well-being in 2018 as the dependent variables, while controlling for the earlier measures in 2014.

\textbf{Trust.} Both generalized and particularized trust were obtained from the CFPS in 2014. Following measures used in previous studies \parencite{churchillTrustSocialNetworks2017,baiSocialTrustPattern2019,kramerIngroupOutgroupTrustBarriers2018}, I assessed particularized trust by the average trust toward family members and neighbors on a 0 (\textit{the lowest}) to 10 (\textit{the highest}) scale. Cronbach's alpha of 0.80 indicated good internal consistency. Unfortunately, I could not include trust toward friends or acquaintances, since they were not covered by the CFPS. Generalized trust was measured by a single survey item on the level of trust toward strangers, using the same 0 to 10 scale.

\textbf{Relational Pathways.} I selected marital satisfaction and interpersonal relationship as the potential mediators through which trust influences couples' subjective well-being. Both indicators were reported by the CFPS in 2018. Marital satisfaction was assessed by a single-item survey question on a scale from 1 (\textit{very dissatisfied}) to 5 (\textit{very satisfied}). Interpersonal relationship was assessed by a single-item survey question on a scale from 1 (\textit{very bad}) to 5 (\textit{very good}).

\textbf{Covariates.} I controlled for several socioeconomic and demographic variables that may confound the relationships between trust and couples' subjective well-being, including age, age squared, number of children, duration of marriage, years of education, and the logarithm of household income per capita, all measured in 2014. Except for couple-level variables (e.g., number of children, duration of marriage, household income), all covariates identify both husbands and wives individually. To minimize biases caused by omitted variables, I also controlled for subjective well-being in 2014. Additionally, to account for contextual influence of trust, I included controls for aggregated levels of generalized and particularized trust by county.

\subsection{Analytical Strategy}

The Actor-Partner Interdependence Model (APIM) was employed to examine the dyadic-level influence of trust, as illustrated in Figure~\ref{fig:APIM}. The APIM has been widely applied to study various types of dyads, including husband-wife and parent-child relationships \parencite{kennyReflectionsActorpartnerInterdependence2018}, due to its capability of separating the actor and partner effects within dyadic interactions. Here, the actor effects, denoted as $\alpha_1$ for husbands and $\alpha_2$ for wives, measure the extent to which individuals' trust affects their own subjective well-being, respectively. The partner effects, $p_1$ and $p_2$, refer to the influence of the husband's trust on the wife's subjective well-being, and vice versa. Both actor and partner effects are estimated while controlling for the intra-couple covariance in trust and subjective well-being, denoted as $c_1$ and $c_2$, respectively.

Following the approach outlined by \textcite{kennyReflectionsActorpartnerInterdependence2018}, I compared two sets of APIMs: one with gender variations and the other without. In the first set of models, all parameters were unconstrained, allowing the actor and partner effects for men ($\alpha_1$ and $p_1$) to differ from those for women ($\alpha_2$ and $p_2$). The second set of models treated husbands and wives as indistinguishable, equating the effects between genders ($\alpha_1=\alpha_2$, $p_1=p_2$). I compared these two sets of models using $\chi^2$-difference tests and the Bayesian Information Criterion (BIC) to determine if gender differences were significant.

To explore the mediating effects of the relational pathways, I conducted mediation analyses using the APIM with Mediation (APIMeM), as depicted in Figure~\ref{fig:APIMeM} \parencite{ledermannAssessingMediationDyadic2011}. The total, direct, and indirect effects were assessed using bootstrap methods. Similar to the main models, gender variations were evaluated by comparing the constrained and unconstrained models.

All models were estimated using the "lavaan" package in R \parencite{rosseelLavaanPackageStructural2012}. Missing data, which all fell below 1\% except household income per capita (4.8\%) and duration of marriage (4.9\%), was missing at random and handled by the Full Information Maximum Likelihood (FIML) method. Compared to imputation, FIML does not require the manual selection of variables and yields essentially equivalent results \parencite{leeComparisonFullInformation2021}. Models are considered of excellent fit if the Comparative Fix Index (CFI) exceeds 0.97, and both the Root Mean Square Error of Approximation (RMSEA) and the Standardized Root Mean Square Residual (SRMR) are below 0.05 \parencite{schermelleh-engelEvaluatingFitStructural2003}. A Monte Carlo power analysis was conducted, with an alpha level of 0.05 and power of 0.80, to ensure that the sample size was sufficient to reliably test the hypotheses and achieve adequate model fit (CIF > 0.95, RMSEA < 0.06). All variables and estimates in the following report have been standardized.
