\chapter{Conclusion}
\label{chap:conclusion}

One of the foremost concerns today is the growing inequality in nearly all aspects of life. This is alarming not only in developed countries, but also in developing countries like China, where various forms of inequality, whether by gender, education, ethnicity, health, or well-being, that were once byproducts of the bygone burgeoning economy, have now become its haunting aftermath \parencite{sicularUrbanRuralIncomeGap2007,xieIncomeInequalityTodays2014,yeungHigherEducationExpansion2013}. Boundaries across social groups are becoming increasingly difficult to cross, and gender relations have backslidden due to the resurgence of patriarchal values \parencite{jiUnequalCareUnequal2017}. Perhaps no other institution has experienced these repercussions more profoundly than families, and none other than families, through evolving patterns of union formation, reproduction, and relationship dynamics, holds greater potential as sources of social change.

The above chapters have illustrated this point through addressing how individuals and their family members' well-being is impacted by three important family-related and relationship-oriented factors: intergenerational mobility, educational sorting in unions, and trust. Specifically, Chapter~\ref{chap:edu-mobility-swb} has shown that individuals and their parents were not necessarily better off when moving upwards, but both generations' well-being was impaired by downward mobility—a phenomenon observed exclusively among only-child families. I highlight, therefore, the plight of mobility—penalties for falling downwards with no returns to moving upwards—that traps both generations in only-child Chinese families.

Chapter~\ref{chap:edu-sorting-swb} has demonstrated that educational sorting in unions has repercussions on individuals' well-being, with variations between genders and across contexts. Hypergamy (women partnering with more educated men) was the least optimal arrangement for both genders, and men were more satisfied with life when partnering upwards. Notably, hypergamy will become even more unfavorable for women's well-being over time, as its well-being disadvantage is exacerbated in contexts where such partnerships are less normative.

In Chapter~\ref{chap:trust-swb}, I focused on the role of trust in shaping individuals' and their spouse's subjective well-being through relational pathways. I differentiated between generalized and particularized trust, and found that particularized trust, through improving marital satisfaction and interpersonal relationships, significantly enhanced individuals' well-being in all dimensions and extended its benefits from husbands to wives, but not vice versa. Generalized trust, on the other hand, had limited effects and did not spill over. These findings suggest that husband-wife dyadic interactions remain gendered and asymmetric, and that particularized trust is the more effective trust type for improving couples' well-being.

Theoretically, these chapters contribute to the literature by underscoring the importance of embracing the perspective of linked lives, that individuals' well-being is not solely determined by his or her own characteristics, but contingent upon their related ones, and therefore, must be studied \textit{in relation to} others and sensitive of the social context.

Methodologically, this dissertation demonstrates the use of the Diagonal Mobility Models to disentangle the effects of various factors within the stratification-generating mechanisms, as well as the use of the Actor-Partner Interdependence Models to distinguish various relational pathways through which trust enhances well-being. These methods are particularly useful in operationalizing the linked lives' perspective.

Future research may address the limitations of these studies, including but not limited to examining the varying outcomes of educational mobility and sorting across the educational spectrum, addressing the contextual characteristics apart from the normative climate, and incorporating the emerging non-conventional families.
