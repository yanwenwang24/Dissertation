\chapter{Conclusion}
\label{chap:conclusion}

One of the foremost concerns today is the growing inequality in nearly all aspects of life. This is alarming not only in developed countries, but also in developing countries like China, where various forms of inequality, whether by gender, education, ethnicity, health, or well-being, that were once byproducts of the bygone burgeoning economy, have now become its haunting aftermath \parencite{sicularUrbanRuralIncomeGap2007,xieIncomeInequalityTodays2014,yeungHigherEducationExpansion2013}. Boundaries across social groups are becoming increasingly difficult to cross, and gender relations have backslidden due to the resurgence of patriarchal values \parencite{jiUnequalCareUnequal2017}. Perhaps no other institution has experienced these repercussions more profoundly than families, and none other than families, through evolving patterns of union formation, reproduction, and relationship dynamics, holds greater potential as sources of social change.

The above chapters have illustrated this point through addressing the well-being outcomes of family-related stratification-generating mechanisms, including intergenerational mobility and educational sorting in unions, and through examining the evolving trends in marital sorting by education. Specifically, Chapter~\ref{chap:edu-mobility-swb} has shown that individuals and their parents were not necessarily better off when moving upwards, but both generations' well-being was impaired by downward mobility—a phenomenon observed exclusively among only-child families. I highlight, therefore, the plight of mobility—penalties for falling downwards with no returns to moving upwards—that traps both generations in only-child Chinese families.

Chapter~\ref{chap:edu-sorting-swb} has shown that educational sorting in unions has repercussions on individuals' well-being, with variations between genders and across contexts. Hypergamy (women partnering with more educated men) was the least optimal arrangement for both genders, and men were more satisfied with life when partnering upwards. Notably, hypergamy will become even more unfavorable for women's well-being over time, as its well-being disadvantage is exacerbated in contexts where such partnerships are less normative.

In Chapter~\ref{chap:explain-edu-sorting}, I explained the trends in marital sorting by education in China, using a decomposition approach. I found that the initial decrease in homogamy among earlier cohorts were driven entirely by educational expansion. For later cohorts, sustained educational expansion became unfavorable for hypergamy, outweighing the opposing effects of a steeper decline in marriage rates among highly educated women. Preferences for homogamy and against heterogamy, especially hypogamy, intensified. Together, these factors explained why educational sorting outcomes in marriage, namely, the rising homogamy, the declining hypergamy, the stagnant hypogamy, have occurred as they have.

Theoretically, these chapters contribute to the literature by bridging the scholarship on social stratification with the one on well-being. They also highlight the importance of incorporating a comparative perspective and the linked lives' perspective in the study of the outcomes of stratification, that is, these outcomes extend beyond the individual to their family members and differ across contexts.

Methodologically, this dissertation underscores the importance of disentangling the effects of various components within the stratification-generating mechanisms. The Diagonal Mobility Models and the decomposition approach are particularly useful for this purpose.

Future research may address the limitations of these studies, including but not limited to examining the varying outcomes of educational mobility and sorting across the educational spectrum, addressing the contextual characteristics apart from the normative climate, and incorporating the emerging non-conventional families.