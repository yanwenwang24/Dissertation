\chapter{Introduction}
\label{chap:introduction}

One of the foremost concerns today is the growing inequality in nearly all aspects of life. This is alarming across the globe and especially in developing countries like China, where various forms of inequality, whether by gender, education, ethnicity, health, or well-being, that were once byproducts of the bygone burgeoning economy, have now become its haunting aftermath \parencite{sicularUrbanRuralIncomeGap2007,xieIncomeInequalityTodays2014,yeungHigherEducationExpansion2013}. Boundaries across social groups are becoming increasingly difficult to cross, and gender relations have backslidden due to the resurgence of patriarchal values \parencite{jiUnequalCareUnequal2017}. Perhaps no other institution has experienced these repercussions more profoundly than families, and none other than families, through evolving patterns of union formation, reproduction, and relationship dynamics, holds greater potential as sources of social change.

Against this backdrop, this dissertation consists of three papers on stratification, family, and subjective well-being. Specifically, I embrace the perspective of linked lives and examine how individuals and their family members' well-being is impacted by three important family-related, stratification-associated, and relationship-oriented factors: intergenerational mobility, educational sorting in unions, and trust.

Chapter~\ref{chap:edu-mobility-swb} addresses how intergenerational mobility affects the well-being of Chinese families. It is the first to extend beyond primary movers and encompass their parents, with an emphasis on family structures and parent-child gender dynamics. Using the Diagonal Mobility Model (DMM) to analyze data from the China Family Panel Studies, I found that net of origin and destination, those moving upwards were not necessarily better off, but both generations' well-being was impaired by downward mobility—a phenomenon observed exclusively among only-child families. Among these parents, mothers with an upwardly mobile daughter reported the highest life satisfaction. These findings highlight the departure from traditional son preferences and the plight of mobility—penalties for falling downwards with no returns to moving upwards—that traps both generations in only-child Chinese families.

Chapter~\ref{chap:edu-sorting-swb} shifts the attention from intergenerational mobility to educational sorting in unions. It is a comparative study that investigates the association between educational sorting \textit{per se} and subjective well-being of heterosexual partners in Europe, with a focus on gender and contextual variations. Using the DMM to analyze data from the European Social Survey, I found that net of status effects, hypergamy (women partnering with more educated men) was associated with lower well-being for both genders, and men were more satisfied with life in hypogamy (partnering with more educated women). These patterns varied across societies, illustrated, for instance, by a hypergamy advantage among men in Southern Europe and women in the Baltic States. Notably, women's well-being disadvantage in hypergamy was exacerbated in contexts where such partnerships were less normative. These findings provide unique insights into the diverse well-being outcomes of educational sorting between genders and across societies, shaped, in part, by societal norms.

While the above chapters have examined status comparisons within families and by extension, stratification, Chapter~\ref{chap:trust-swb} focuses on trust as a critical determinant of couples' subjective well-being. Viewing both trust and well-being as multidimensional and relational constructs, this study explores how generalized and particularized trust, both measured using survey items sensitive to the "trust radius" in China, influences individuals' and their married spouse's subjective well-being across multiple dimensions—life satisfaction, happiness, and depression—via relational pathways. Lagged Actor-Partner Interdependence Models with Mediation were employed to analyze a sample of 6,964 pairs of married couples selected from the China Family Panel Studies in 2014 and 2018. Results show that particularized trust, through improving marital satisfaction and interpersonal relationships, significantly enhanced individuals' well-being in all dimensions and extended its benefits from husbands to wives, but not vice versa. Generalized trust had limited effects and did not spill over. These findings highlight the gendered dynamics within husband-wife dyadic interactions and underscore the importance of relational contexts in shaping the well-being implications of trust.
