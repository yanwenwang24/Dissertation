\section{Theoretical Background}
\label{sec:ch3-theoretical-background}

\subsection{Educational Sorting and Subjective Well-Being}

This study focuses on educational sorting for two main reasons. First, as a long-standing predictor of individuals' future earning potential, education is one of the first and most important filters in mate selection \parencite{kalmijnIntermarriageHomogamyCauses1998,schwartzTrendsEducationalAssortative2005}. This holds true for both genders, as women's economic roles in relationships have largely converged with those of men \parencite{oppenheimerTheoryMarriageTiming1988}. Second, individuals often obtain their highest education before union formation, making education a convenient indicator of status comparison between partners, especially when couples' other characteristics at the time of union formation are not often available.

Previous studies have well documented the well-being gaps across different educational pairings and proposed, in general, three competing propositions: the homogamy advantage, returns to status attainment, and gender roles. The first hypothesis views homogamy as the optimal arrangement for couples’ subjective well-being \parencite{kalmijnIntermarriageHomogamyCauses1998}. Here, similar education levels serve as proxies for compatibility in socioeconomic resources and intangible qualities such as values, beliefs, tastes, and knowledge structures. Research has shown that couples comparable in socioeconomic resources tend to coordinate well in family life and are more likely to adopt an egalitarian division of labor, leading to reduced stress, fewer depressive symptoms, and higher relationship satisfaction \parencite{nitscheCouplesIdeologicalPairings2023}. Additionally, sharing similar attitudes and beliefs fosters mutual affirmation \parencite{berlamontCoupleSimilarityEmpathic2024}; having common interests and tastes enhances the quality of leisure time spent together \parencite{beckerSelectionAlignmentTheir2010}; and possessing similar knowledge structures facilitates meaningful conversations and mutual understanding \parencite{gauntCoupleSimilarityMarital2006}, all of which contribute to positive well-being.

In Europe, empirical studies have found evidence supporting the homogamy advantage. In the Netherlands, \textcite{keizerAreEqualsHappier2015} found that couples' dissimilarity in socioeconomic characteristics was associated with lower life satisfaction for both partners. In Germany, \textcite{stutzerDoesMarriageMake2006}, using a longitudinal dataset spanning 17 years, found that couples with educational differences above the median experienced lower life satisfaction after marriage. Other European studies suggested lower divorce risks for homogamous couples \parencite{beckerSelectionAlignmentTheir2010,kalmijnUnionDisruptionNetherlands2003}, which are often associated with positive well-being.

Based on these rationales, I propose the first hypothesis of homogamy advantage:

\begin{quote}
    \textbf{H1.} Compared to homogamy, heterogamy is negatively associated with individuals' subjective well-being.
\end{quote}

The status attainment proposition underscores partnership as a crucial channel for social mobility \parencite{taylorUtilityEducationAttractiveness1976}. In this regard, while education homogamy symbolizes status maintenance, heterogamy represents a step up or down on the social ladder for both partners, where they engage in status comparison and assess their gains or losses from the relationship \parencite{pearlinStatusInequalityStress1975,zhaoPartnersEducationalPairings2022}. Therefore, those who partner upwards in education—men in hypogamy and women in hypergamy—gain access to the occupational prestige, economic resources, and social capital associated with their partner's higher education, which yield positive well-being returns. Conversely, partnering downwards represents a failure to leverage partnerships as effectively for status attainment, potentially leading to a decline in well-being.

Several studies have documented the well-being returns or losses associated with status attainment through marriage. In the U.S., \textcite{pearlinStatusInequalityStress1975} found that both men and women who married downwards in education suffered greater emotional distress, disrupting the reciprocity, affection, and value consensus within marriage. In China, \textcite{zhaoPartnersEducationalPairings2022} demonstrated that satisfaction with status attainment was the most fitting explanation, as both genders marrying upwards were happier than their homogamous counterparts. These findings, albeit outside the European context, lay the groundwork for the second hypothesis:

\begin{quote}
    \textbf{H2.} Compared to homogamy, individuals' subjective well-being is positively associated with partnering upwards in education and negatively related to partnering downwards.
\end{quote}

Still, the significance of gender roles in mate selection behaviors and outcomes cannot be overlooked. From the “doing gender” perspective \parencite{westDoingGender1987}, intimate relationships are the primary locus where gender roles are enacted and reinforced through within-couple power dynamics, and in this case, through the educational mismatch in heterogamy.

In the event that traditional gender roles, in which women are assigned to subordinated positions, prevail, hypogamy would pose significant challenges for both genders \parencite{connellHegemonicMasculinityRethinking2005,kalmijnUnionDisruptionNetherlands2003}. This relationship configuration, although providing men with a channel for status attainment, may jeopardize their masculine identity, a key determinant of men's well-being \parencite{courtenayConstructionsMasculinityTheir2000}. Women face the dual detriments of partnering downwards and deviating from male-dominated gender roles \parencite{potarcaAreWomenHypogamous2022}. For these reasons, couples in hypogamy often experience escalated conflicts \parencite{centersConjugalPowerStructure1971}, relationship dissatisfaction \parencite{beanFamilismMaritalSatisfaction1977}, and heightened risks of divorce \parencite{kalmijnUnionDisruptionNetherlands2003,theunisHisHerEducation2018}. These negative consequences are more acute for women, who suffer, in extreme cases, emotional and physical abuse \parencite{kaukinenStatusCompatibilityPhysical2004}.

Unlike hypogamy, hypergamous relationships typically receive stronger social approval for aligning with male-dominated gender roles. They may nonetheless yield diverging well-being implications between genders. Men, albeit partnering downwards, may benefit from strengthened masculinity and a highly gendered—hence, according to \textcite{beckerTheoryMarriagePart1973,beckerTheoryMarriagePart1974}, efficient—division of labor. Women, despite partnering upwards and receiving less social disapproval, may find themselves inundated with patriarchal expectations. With less bargaining power, they have less say in financial decision-making and take more housework and childcare burdens \parencite{klesmentWomensRelativeResources2022,millerEducationalAssortativeMating2020}. If the stress and sacrifices outweigh the benefits of partnering upwards, hypergamy may adversely impact women's well-being.

Although gender relations have become more egalitarian, traditional gender roles remain powerful in shaping relationship outcomes. A recent study on nine European countries demonstrated that the disadvantage in life satisfaction experienced by both genders in female-breadwinning relationships was fairly universal, even in countries with weak institutional support for the male-breadwinner model \parencite{kowalewskaFemalebreadwinnerWellbeingPenalty2023}. In Hungary, \textcite{hajduIntraCoupleIncomeDistribution2018} found that women's higher income relative to their male partners was associated with lower subjective well-being for both genders. Outside Europe, \textcite{booysenIntermarriageSubjectiveSocial2022} highlighted the gender-asymmetric impact of hypergamy: hypergamous couples exhibited the largest life satisfaction gap in favor of men. Based on these findings, I propose the following hypotheses:

\begin{quote}
    \textbf{H3a.} Compared to homogamy, hypogamy is negatively associated with individuals' subjective well-being, with the adverse impact being more pronounced for women.

    \textbf{H3b.} Compared to homogamy, hypergamy is positively associated with men's subjective well-being but negatively related to women's.
\end{quote}

\subsection{Adding a Contextual Dimension to the Outcomes of Educational Sorting}

Apart from revisiting the aforementioned hypotheses, I extend the literature by investigating how individuals, as social beings, experience the well-being implications of educational sorting differently across contexts. To the best of my knowledge, three studies are notable exceptions that explored the nuanced interplay between individual mate selection and contextual backgrounds. \textcite{schwartzReversalGenderGap2014} found that higher divorce risks associated with hypogamy disappeared among younger U.S. cohorts, potentially due to a shift towards more egalitarian gender role attitudes. In Belgium, \textcite{theunisHisHerEducation2018} examined spatial variations and found that in communities where hypogamy was more common, its risks of divorce diminished. \textcite{potarcaAreWomenHypogamous2022}, in their study of 27 European countries, found that the mental health disadvantages faced by married women in hypogamy were mitigated in contexts with greater female empowerment. Despite these valuable insights, much remains unexplored regarding whether and how the well-being outcomes of educational sorting are shaped by the macro context. Therefore, this study builds on these works by taking a country-clustering approach and examining the conditioning role of the normative climate.

\subsubsection{Variation Across Country Clusters}

Home to 44 sovereign states, Europe is highly heterogeneous, with a rich tapestry of cultures, histories, politics, and economies, ranging from individualistic to familial cultures, from long-established democracies to post-socialist states, and from capitalist economies to universal welfare \parencite{castlesWorldsFamiliesRegimes2008,esping-andersenThreeWorldsWelfare1990,ferreraSouthernModelWelfare1996}. These diversities are associated with variations in not only the dynamic patterns of educational sorting, but also, of primary interest to this study, their well-being outcomes. Due to limited sample size for each country, I grouped the 29 countries under investigation into six clusters, based on welfare regime typologies, social and gender inequality, and educational sorting patterns \parencite{dehauwReversedGenderGap2017,domanskiEducationalHomogamy222007}.

According to \citeauthor{esping-andersenThreeWorldsWelfare1990}'s (\citeyear{esping-andersenThreeWorldsWelfare1990}) welfare regime typologies, the Nordics (e.g., Denmark, Finland) have their origins in social democratic regimes characterized by de-commodification, cross-class solidarity, and universal welfare \parencite{artsModelsWelfareState2010}. The emphasis on resource redistribution leads to less rigid barriers across social and educational groups. Gender relations are relatively egalitarian, evidenced by women's significant educational advantage over men and narrow gender pay gaps. In this study's sample (detailed later), Nordic countries have some of the highest heterogamy rates (e.g., 55.53\% in Iceland, 48.42\% in Sweden), especially hypogamy.

By contrast, Southern European countries (e.g., Italy, Portugal), later incorporated in \citeauthor{esping-andersenThreeWorldsWelfare1990}'s (\citeyear{esping-andersenThreeWorldsWelfare1990}) framework as the Mediterranean regimes, are characterized by traditional family values, with welfare systems relying on kinship solidarity as the locus of care and support \parencite{ferreraSouthernModelWelfare1996}. Compared to Nordic countries, income and social inequality is more pronounced in Southern Europe, and gender relations, as shown by various indicators (e.g., gender gap in housework time, married women's labor force participation rates), remain relatively unequal \parencite{gonzalezGenderInequalitiesSouthern2000,renGenderInequalitiesWork2023}. Homogamy rates in the sample range from 61.23\% in Spain to 73.85\% in Portugal, some of the highest in Europe.

Continental European (e.g., France, Germany) and Anglo-Saxon countries (e.g., Ireland, the U.K.) are two clusters with less redistributive welfare compared to Nordic countries \parencite{esping-andersenThreeWorldsWelfare1990,esping-andersenWelfareRegimesSocial2015}. The differences between the two are that the conservative models in Continental Europe feature strong state intervention and insurance-based benefits linked to employment, whereas Anglo-Saxon countries follow a more liberal path, offering, except for universal healthcare, mostly residual-based welfare with means-tested assistance \parencite{artsModelsWelfareState2010}. Compared to other regions, women in Continental Europe, and to a lesser extent in Anglo-Saxon countries, have less educational advantage over men \parencite{eratEducationalAssortativeMating2021}. In some countries with institutionally supported male-breadwinner models (e.g., Germany, Switzerland), women even face educational disadvantages, and those among younger cohorts are more likely to be in hypergamous relationships than in hypogamous ones.

Central \& Eastern European countries (e.g., Bulgaria, Slovakia) form a distinct cluster of post-communist states \parencite{castlesWorldsFamiliesRegimes2008}. Their transitions to market economies are marked by rudimentary welfare and the increasing saliency of education in reproducing social stratification \parencite{fengerWelfareRegimesCentral2007,mullerEducationYouthIntegration2005}. The Baltic States (e.g., Estonia, Lithuania) stand out from other post-socialist countries, as they lean towards a more liberal path with fast-growing social inequality and, quite surprisingly, high rates of heterogamy resembling those in the Nordics \parencite{aidukaiteTransformationWelfareSystems2009}. In contrast, other Central \& Eastern European countries exhibited some of the highest prevalence of homogamy in Europe, led by Slovakia (74.85\%), Bulgaria (72.07\%), and the Czech Republic (66.82\%).

The six-cluster approach is, by all means, not the definitive way of capturing all the nuanced differences across countries. It is rather a crude classification that is congruent with previous comparative studies on Europe (e.g., \cite{castlesWorldsFamiliesRegimes2008,schuckDoesIntergenerationalEducational2018}) and lays the framework for understanding the potentially diverse well-being outcomes of educational sorting. I, therefore, cannot raise definitive hypotheses but make the following speculations. Educational homogamy is likely more advantageous in liberal regimes with welfare policies designed to reinforce, rather than reduce, inequality, such as Anglo-Saxon countries and the Baltic States. Partnering upwards in education is expected to yield greater well-being returns in contexts where it is crucial for upward mobility, such as Central \& Eastern Europe. Gender roles, manifested, for instance, in the relative advantage of hypergamy for men, are more influential in countries with traditional gender values, such as Continental and Southern Europe. Lastly, educational differences between partners may have little impact on both genders' well-being in the Nordics with universal welfare and fairly egalitarian gender relations.

\subsubsection{The Normative Climate of Educational Sorting}

To gain further insights into contextual variations, I focus on the normative climate of educational sorting varying not only by region but also across periods—a temporal dimension not addressed by the country-clustering approach. According to \textcite{cherlinDeinstitutionalizationAmericanMarriage2004}, the normative climate refers to the extent to which a particular type of relationship has been strengthened and reinforced, thereby becoming institutionalized and forming part of the social norms that influence individuals' mate selection behaviors and outcomes.

The normative climate may influence mate selection outcomes via two competing mechanisms. First, in contexts where a certain type of relationship is highly normative, individuals who deviate from such norms may face strong social disapproval \parencite{keizerAreEqualsHappier2015,soonsMarriageMoreCohabitation2009}. They often experience greater stress, reduced support, ambiguous feelings about the future, and a decline in relationship and life satisfaction. Conversely, those who conform to social norms typically face fewer challenges. Comparative studies on marriage premiums in well-being have supported the social disapproval hypothesis. For instance, \textcite{soonsMarriageMoreCohabitation2009} found that in European countries where cohabitation was more common, the gap in well-being between married and cohabiting partners was significantly smaller. Likewise, \textcite{kalmijnCountryDifferencesEffects2010} noted that the negative impact of divorce on well-being was mitigated in contexts with high divorce rates.

The alternative scenario emphasizes individuals' self-selection into certain relationships. For instance, women’s selection into hypogamy, especially in contexts where hypergamy is highly normative and men possess an educational advantage over women, is not likely driven by a structural shortage of higher-educated men, but rather by a strong sense of personal conviction \parencite{potarcaAreWomenHypogamous2022}. These individuals form a group of non-conformists for whom normative pressures are merely a testimony to their autonomy. In Europe, \textcite{potarcaAreWomenHypogamous2022} found that highly-educated women had fewer depressive symptoms associated with hypogamy in contexts where hypergamy was more common.

I test these competing mechanisms against two crucial aspects of the normative climate: the dominance of homogamy over heterogamy, and the dynamics between hypergamy and hypogamy, both of which vary considerably across countries and periods in patterns that may not align well with predefined country clusters. I propose the following hypotheses:

\begin{quote}
    \textbf{H4a.} A certain type of educational sorting yields greater positive influence (or exerts less negative influence) on individuals' subjective well-being in contexts (country-period units) where it is more normative.

    \textbf{H4b.} A certain type of educational sorting yields greater positive influence (or exerts less negative influence) on individuals' subjective well-being in contexts where it is less normative.
\end{quote}

\subsection{Isolating Educational Difference from Couples' Education}

It is empirically and theoretically important to address the wellbeing implications of educational sorting per se—the difference in education between partners, independent of individuals' and their partner's absolute levels of education \parencite{eeckhautAnalysingEffectEducational2013,eeckhautEducationalHeterogamyDivision2014}. The returns to education vary across genders and contexts \parencite{horieReturnsSchoolingEuropean2023}, which, if not controlled for, may obstruct scholars from uncovering the sociopsychological realm of educational sorting, that is, the subjective perception and experience of the symbolic spousal difference in education.

Previous studies have taken two main approaches: difference and compound measures, each with shortcomings (see a detailed review by \cite{eeckhautAnalysingEffectEducational2013}). The first approach quantifies spousal dissimilarity numerically or categorically. It is, however, statistically infeasible to vary one component (e.g., educational difference) while holding the other two (e.g., each partner's education) constant in linear regressions. As a compromise, scholars often model the effects of educational difference while controlling for only one partner's education, leading to estimations confounded by the omitted factor (e.g., \cite{gongDoesStatusInconsistency2007,tynesEducationalHeterogamyMarital1990}). For example, \textcite{gongDoesStatusInconsistency2007} possibly overestimated the benefits of hypogamy by omitting the wife's education.

The other approach using compound measures estimates the well-being gaps across all possible educational pairings by cross-tabulating individuals' and their partner's educational levels (e.g., \cite{jalovaaraJointEffectsMarriage2003}). This approach requires very large sample sizes to sufficiently represent each educational combination, and it does not effectively distinguish the effects of educational difference from those of education. Scholars have also proposed interaction models that estimate the respondent's education, their partner's education, and the interaction terms between the two—a mathematical equivalent to the former but with the advantage of testing whether the interaction terms add to the main effects of education (e.g., \cite{vannoyRelativeSocioeconomicStatus2001}). However, a significant interaction term may not necessarily indicate the effects of educational difference, since this model, by treating educational difference as a linear combination of two education components, has already “incorporated such an effect within its own variance” (\cite[p.332]{hopeModelsStatusInconsistency1975}; \cite{sobelDiagonalMobilityModels1981,sobelSocialMobilityFertility1985}). Moreover, the interaction model compares the outcomes of heterogenous couples with those of men's educational reference group and those of women's educational reference group, rather than with homogamous couples who, from a sociological perspective, form the core of the educational group and are thus more appropriate reference groups \parencite{sobelDiagonalMobilityModels1981,sobelSocialMobilityFertility1985}.

In this study, I employ the Diagonal Mobility Model (DMM) to estimate the influence of educational difference between partners, independent of each partner's education. Originally proposed by \textcite{sobelDiagonalMobilityModels1981}, the DMM tackles the methodological challenge long haunting mobility research, that is, similar to the aforementioned issues, the isolation of mobility per se from origin and destination. Despite the recognition of the DMM as the most robust method, so far it has been applied by only a few studies to explore various outcomes of status difference between partners, including the division of labor in Belgium and Sweden \parencite{eeckhautEducationalHeterogamyDivision2014}, fertility decisions in the U.S. \parencite{sorensonHusbandsWivesCharacteristics1989}, and subjective well-being in China \parencite{zhaoPartnersEducationalPairings2022}.

The DMM is founded on the sociological assumption that partners sharing similar educational levels (e.g., homogamy) form the core of various educational groups, with their outcomes (e.g., subjective wellbeing) representing the characteristics of their respective group \parencite{sobelDiagonalMobilityModels1981}. It treats homogamous partners as the diagonal reference group and places heterogamous ones off the diagonal, between their own and their partner's educational levels. This means that the well-being of individuals in heterogamy is modeled as a weighted sum of the well-being of homogamous partners at the corresponding educational levels. The weights, summing up to one, reflect the relative influence of individuals' and their partner's education on subjective well-being. The effects of educational difference, then, can be estimated over and above these influences. The baseline DMM is specified as follows:

\begin{equation}
    \label{eq:dmm-chapter3}
    \text{Y}_{ijk} = p \mu_{ii} + (1-p) \mu_{jj} + \sum \beta X_{kl} + \epsilon_{ijk}, 0 \leq p \leq 1
\end{equation}

where, $Y_{ijk}$ is the subjective well-being of individual $k$ whose education is at level $i$ and whose partner's education is at level $j$, $\mu_{ii}$ and $\mu_{jj}$ are the subjective well-being of homogamous partners at educational levels $i$ and $j$, respectively, and $\sum \beta X_{kl}$ is a vector of covariates. Weight parameters $p$ and $(1-p)$ capture the relative influence of individuals' and their partner's education, respectively.

The effects of educational sorting can be assessed by adding additional variables into the baseline model, which is given by:

\begin{equation}
    \label{eq:dmm_heter}
    \text{Y}_{ijk} = p \mu_{ii} + (1-p) \mu_{jj} + \gamma_{1} \text{Heter}_{k} + \sum \beta X_{kl} + \epsilon_{ijk}, 0 \leq p \leq 1
\end{equation}

\begin{equation}
    \label{eq:dmm_hyper}
    \text{Y}_{ijk} = p \mu_{ii} + (1-p) \mu_{jj} + \gamma_{2} \text{Hyper}_{k} + \gamma_{3} \text{Hypo}_{k} + \sum \beta X_{kl} + \epsilon_{ijk}, 0 \leq p \leq 1
\end{equation}

in which, $\gamma_{1}$, $\gamma_{2}$, and $\gamma_{3}$ capture the respective effects of heterogamy, hypergamy, and hypogamy on subjective well-being, net of partners' education and covariates.

The DMM is estimated using maximum likelihood estimation. To ensure the weight parameter p remains within [0,1], a logistic transformation is employed: $p = exp(\alpha)/(1+exp(\alpha))$, where $\alpha$ is the unconstrained parameter estimated in the model. This specification offers several advantages over conventional approaches, including a direct operationalization of the theoretical premise that homogamous couples represent the core characteristics of each educational group, a parsimonious and interpretable measure of whether one's own or partner's education level is more influential, and the estimation of additional effects of educational difference while maintaining the theoretical structure of the model. Please see \citeauthor{eeckhautAnalysingEffectEducational2013}'s (\citeyear{eeckhautAnalysingEffectEducational2013}) comprehensive review of various methodological approaches to studying the outcomes of educational sorting.

Thus far, I have detailed the use of the DMM in estimating the associations between educational sorting and subjective well-being. Next, I will introduce the sample population, individual- and contextual-level measures, and analytical procedures.
