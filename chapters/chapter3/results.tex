\section{Results}
\label{sec:ch3-results}

\subsection{Descriptive Statistics}

Table~\ref{tab:descriptive_stats_weighted} summarizes the weighted sample characteristics. Women, accounting for 50.72\% of the sample, reported significantly higher life satisfaction ($M$=6.87, $SD$=2.27) compared to men (M=6.80, $SD$=2.27). At the time of the survey, the average age of respondents was 45.74 years old ($SD$=10.98), with men being slightly older. The average household size was 3.28 ($SD$=1.14). Respondents, on average, had 1.11 children ($SD$=0.99) living together, and 22.63\% had at least one child under age 6. Around 10\% of respondents identified themselves as immigrants, and 7.09\% considered themselves minorities. Around one-tenth had divorced before, 6.10\% were unemployed, and only 28.12\% reported living comfortably on present income.

Among all respondents, 24.10\% had lower secondary education or less, 37.12\% had upper secondary education, 10.17\% completed non-tertiary education, and 28.61\% had tertiary education. By the most recent survey year of each country, the gender educational gap had reversed, with women having an educational advantage in all countries except Germany, Italy, and Switzerland.

\subsection{Trends in Educational Sorting}

Table~\ref{tab:edu_sorting} shows educational sorting patterns. Homogamy remained the most dominant pattern, accounting for 57.16\% of all relationships, whereas hypergamy and hypogamy accounted for 20.86\% and 21.97\%, respectively. As suggested by \textcite{sobelSocialMobilityFertility1985}, in the DMMs, a sufficient sample size on the diagonal (e.g., homogamy) is necessary for representing the characteristics of each social position. In this case, the widespread presence of homogamy across the educational spectrum likely satisfies this prerequisite.

Figure~\ref{fig:trends_edu_sorting} illustrates the changing landscape of educational sorting o comes across regions and cohorts. Across Europe, homogamy was most dominant in Southern and Central \& Eastern Europe, particularly in Slovakia (74.85\%), Portugal (73.85\%), Bulgaria (72.07\%), and the Czechia Republic (66.82\%), and least prevalent in Nordic countries and the Baltic States, such as Iceland (44.47\%), Estonia (48.91\%), and Sweden (51.58\%). Southern Europe stood out for a significant decline in homogamy, which was uncommon in other regions. Moreover, amidst the reversal of gender gaps in education, the most dramatic changes occurred in the decline of hypergamy and the rise of hypogamy. The proportion of hypergamous relationships dwindled from 23.36\% among those born in the 1950s to 16.62\% among those born in 1980 or later, accompanied by a corresponding increase in hypogamy. By the most recent survey year for each country, hypogamy had surpassed hypergamy, becoming the second most common type in 25 out of 29 countries, except for Austria, Germany, Greece, and Switzerland.

\subsection{Pooled Analysis}

Table~\ref{tab:dmm_pooled} presents the results of the DMMs on life satisfaction of pooled respondents, organized into four parts. The weight parameters (Part \RNum{1}) represent the relative influence of respondents' and their partner's education on life satisfaction, with these weights summing to 1. The educational sorting indicators (Part \RNum{2}) capture the effects of educational differences between partners (heterogamy, hypergamy, and hypogamy) compared to homogamy, net of each partner's education and covariates. The diagonal intercepts (Part \RNum{3}) represent the life satisfaction levels of homogamous couples at each educational level, serving as reference points from which the well-being of heterogamous couples is estimated. Finally, the covariates (Part \RNum{4}) control for various demographic and socioeconomic characteristics.

Baseline models include, besides covariates, only the weight parameters of respondents' and their partner's education. For men, their own education (0.62) was more influential to life satisfaction than their female partner's (0.38). In contrast, women's life satisfaction was less affected by their own education (0.33) than by their male partner's (0.67). These gendered patterns suggest the persistence of male dominance in heterosexual relationships.

Subsequent models estimate the effects of educational sorting, over and above each partner's education. Results of Model 2 show that women's life satisfaction, but not men's, was negatively associated with heterogamy. The gender difference, as indicated by the Z-statistic of 1.63, was not statistically significant.

To explore the variations across subpopulations within heterogamy, I differentiated heterogamy by hypergamy and hypogamy in Model 3, which led to significant improvements in model fits for both genders. Men were more satisfied with life when partnering upwards in education (hypogamy) but less so if partnering downwards (hypergamy). Among women, the homogamy advantage was attributed to lower life satisfaction associated with hypergamy, but not to those in hypogamy. Additionally, men in hypogamy reported marginally higher life satisfaction than their female counterparts.

Diagonal intercepts from the baseline model (which changed little in subsequent models) reveal a distinct educational gradient in life satisfaction among homogamous partners: from 7.59 at the lowest level to 7.62, 7.89, 7.99 at the highest for men, and similarly, but all at higher levels, for women. Both genders were less satisfied with life if they were older, cohabiting, divorced before, immigrants, minorities, unemployed, or unable to live comfortably on present income. Men's life satisfaction, not women's, was positively associated with a larger household size and negatively related to the number of children. Both genders' life satisfaction was positively associated with the presence of children under age 6.

In sum, the pooled analyses indicate a hypergamy disadvantage in life satisfaction for both genders, with hypogamy being the optimal relationship for men's well-being. Thus, these findings validate the homogamy advantage hypothesis (H1) for women and align men with the status attainment proposition (H3).

\subsection{Contextual Variations}

\subsubsection{Variation Across Country Clusters}

To explore contextual variations, I repeated the above analyses on samples further stratified by country cluster. The results are visualized in Figure~\ref{fig:dmm_region}.

As shown by baseline models, the relative influence of partners' education differed across clusters. In Anglo-Saxon countries, the Baltic States, and Continental Europe, individuals' education was as influential to their life satisfaction as their partner's. In contrast, for both genders in Central \& Eastern Europe, and women in Southern Europe, the male partner's education had the dominant influence. Nordic countries stood out as the only cluster where respondents' life satisfaction was more influenced by their partner's education than by their own.

In Central \& Eastern Europe, where education has become increasingly crucial for status attainment during post-socialist transitions \parencite{mullerEducationYouthIntegration2005}, results resemble those from pooled analyses. In line with the status attainment hypothesis, men's life satisfaction was positively associated with partnering upwards but negatively related to partnering downwards. Women, on the other hand, reported lower life satisfaction in hypergamy.

The hypergamy disadvantage was also observed among women in Anglo-Saxon countries with liberal regimes and welfare systems designed to maintain, rather than reduce, inequality \parencite{esping-andersenThreeWorldsWelfare1990,esping-andersenWelfareRegimesSocial2015}. Women were less satisfied with life compared to not only those in homogamy but also their male counterparts in hypergamy ($Z$ = 3.96), suggesting a significant gender gap in life satisfaction in favor of men. Men's life satisfaction was not linked with educational sorting.

The status attainment hypothesis was also in line with findings from the Baltic States, on the part of women. Their life satisfaction was negatively associated with partnering downwards but positively related to partnering upwards, though the latter association was marginally significant. Educational sorting had little to no impact on men's life satisfaction. Gender differences in the well-being outcomes of both hypergamy and hypogamy were marginally significant.

Only in Southern Europe with strong familism did I find the advantage of heterogamy over homogamy: men were most satisfied with life in hypergamous relationships. Educational sorting had little impact on women's life satisfaction, with no significant gender differences.

In other regions, namely, Continental European and Nordic countries, the inclusion of assortative mating indicators did not lead to any significant improvement in model fits. Assortative mating has little, or at best, a weak impact on well-being in either region.

Thus far, I have demonstrated the diverging well-being outcomes of assortative mating across Europe: the homogamy advantage over hypergamy among women in Anglo-Saxon and Central \& Eastern European countries, the returns to status attainment for men in Central \& Eastern Europe and women in the Baltic States, the hypergamy advantage for men in Southern Europe, and weak or negligible effects of educational sorting in Continental European and Nordic countries.

\subsubsection{The Normative Climate}

Following the clustering approach, I further looked into the conditioning role of the normative climate in shaping the well-being outcomes of educational sorting, focusing specifically on two dimensions: the dominance of homogamy over heterogamy, and the dynamics between hypergamy and hypogamy.

Before fitting the model, I confirmed that the H-indexes of homogamy and hypergamy were not significantly correlated ($\rho$=0.08). Thus, multicollinearity is unlikely to be a concern. Additionally, the inclusion of both measures did not lead to any significant change in the coefficients of educational sorting indicators, suggesting that findings from previous analyses—the returns to status attainment for men and the homogamy advantage for women—hold true.

Results in Table~\ref{tab:dmm_inter} show that both contextual measures—the dominance of homogamy over heterogamy and that of hypergamy over hypogamy—were negatively associated with individuals' life satisfaction. Moreover, the negative association between hypergamy and women's life satisfaction was attenuated in contexts where hypergamy was highly normative. Assuming that homogamy is equally prevalent as heterogamy, women's life satisfaction would be positively associated with hypergamy if it accounts for more than 72.94\% of all heterogamous relationships. Meanwhile, those in hypogamy or homogamy did not receive a further increase in well-being regardless of how prevalent hypogamy or homogamy has become, and the returns to status attainment for men were not moderated by the normative climate. Therefore, the social disapproval hypothesis (H4a) was validated on the part of women.

\subsection{Auxiliary Analysis}

To address the potential selection bias of individuals in cohabitation vs. marriage \parencite{dominguez-folguerasCohabitationMoreEgalitarian2013,esteveEducationalHomogamyGap2013,soonsMarriageMoreCohabitation2009}, I repeated the main analyses on married respondents only. Results, summarized in Table~\ref{app:tab:dmm_married}, show a homogamy advantage over hypergamy for both genders. Consistent with earlier analyses, women's disadvantage in hypergamy was exacerbated in contexts where such relationships were less normative. One notable difference, however, was the marginally significant association between hypogamy and married men's life satisfaction, suggesting that the well-being returns to partnering upwards were less pronounced among men in marriage than those in cohabitation.

Given the selection out of marriage \parencite{kalmijnUnionDisruptionNetherlands2003,theunisHisHerEducation2018}, I also conducted analyses on respondents who had never been divorced (see Table~\ref{app:tab:dmm_never_divorced}). Results largely replicated those of the main analyses. Men's life satisfaction was negatively associated with hypergamy but marginally and positively related to hypogamy. Among women who had never divorced, the hypergamy disadvantage was less pronounced but similarly mitigated in contexts where hypergamy was highly normative. Additionally, women in this subsample reported significantly lower life satisfaction in hypogamous relationships, aligning them closely with the homogamy advantage hypothesis.

Analyses using alternative educational classifications (three- and five-level), summarized in Tables~\ref{app:tab:dmm_3level},~\ref{app:tab:dmm_5level},~\ref{app:tab:dmm_3level_inter}, and~\ref{app:tab:dmm_5level_inter} corroborated the main findings: educational hypergamy was associated with lower subjective well-being for both men and women, while men reported higher life satisfaction in hypogamous relationships. The contextual moderation of women's hypergamy disadvantage remained robust, with their well-being deficit being more pronounced in societies where such relationships are less normative. Using the finer five-level educational classification revealed an additional pattern consistent with the social disapproval hypothesis: men's well-being advantage in hypogamous relationships was attenuated in contexts where such partnerships deviate more strongly from social norms.
