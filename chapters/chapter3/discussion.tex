\section{Discussion}
\label{sec:ch3-discussion}

Educational sorting between heterosexual partners has far-reaching implications for social stratification and individuals' subjective wellbeing \parencite{eikaEducationalAssortativeMating2019,keizerAreEqualsHappier2015,potarcaAreWomenHypogamous2022,zhaoPartnersEducationalPairings2022}. Set against the changing landscape and vast diversity of educational pairings \parencite{dehauwReversedGenderGap2017,eratEducationalAssortativeMating2021,esteveEndHypergamyGlobal2016}, this study examines individuals' subjective experience of educational differences between partners, with an emphasis on contextual variations in Europe. Using the DMM to analyze samples pooled from 29 European countries, I found that net of each partner's education, hypergamy was the least optimal configuration for both genders' well-being, and additionally, men reported higher life satisfaction in hypogamous partnerships.

There are, however, heterogeneities across European societies, which provide this study with an excellent field for understanding how individuals, as social beings, experience educational sorting differently across contexts. Initially, I anticipated a greater homogamy advantage, for the sake of status maintenance, in societies with welfare designed to reinforce, rather than reduce, inequality; greater returns to status attainment in post-socialist states where education was particularly important for upward mobility; and the significance of gender roles in societies with less female empowerment. The results diverged, yet not far, from these expectations. For women, hypergamy was the relationship configuration least favorable for well-being in Anglo-Saxon and Central \& Eastern European countries, but in contrast, the most favorable one in the Baltic States, characterized by a more liberal path with fast-growing inequality compared to other post-socialist states and educational sorting patterns resembling those of the Nordics. For men, educational difference between partners had little impact in most regions, except in Central \& Eastern Europe, where education had been crucial for upward mobility during transitions to a market economy, and accordingly, men's life satisfaction was positively associated with partnering upwards but negatively related to partnering downwards, as well as in Southern Europe with strong familism where evidence pointed to a hypergamy advantage. Absent definitive evidence about what contextual factors matter the most, these findings suggest the importance of considering welfare regimes, social inequality, gender relations, and demographic patterns, all highly diverse across European societies and altogether underlying individuals' subjective experience of educational sorting.

Beyond geographical regions, this study further explored the conditioning role of the normative climate across countries and periods. The homogamy advantage over hypergamy among women was attenuated in contexts where hypergamy was highly normative over hypogamy, suggesting the benefits of adhering to social norms. However, those in hypogamy or homogamy did not receive a further increase in wellbeing, regardless of how prevalent the relationship arrangement had become, and the normative climate had little impact on men's wellbeing outcomes of educational sorting. With these findings, this study highlights the nuanced gender differences in the interplay between individual mate selection and societal norms, and the roles of sociocultural, economic, and demographic backgrounds in shaping the subjective experience of educational difference between partners. More importantly, it sends the timely message that hypergamy, already the least optimal relationship configuration for both genders despite its conformity to male-dominated gender roles, may become even more unfavorable for women over time, as with sustained advancement in women's education and female empowerment, a further decline in hypergamy rates is likely in the foreseeable future.

Although this study contributes to the literature on educational sorting and subjective well-being, it is not without limitations. First, the normative climate, albeit important, is not the sole contextual factor affecting individuals' subjective experience of educational sorting. This is a methodological challenge common in comparative studies, as multiple contextual measures are often highly correlated with each other. In this case, the normative climate of hypergamy is correlated with women's educational advantage over men and other socioeconomic and cultural backgrounds \parencite{dehauwReversedGenderGap2017,esteveEndHypergamyGlobal2016}. This is partly why some findings from the country-clustering approach did not align with those from moderation analyses, such as the hypergamy advantage among women in the Baltic States, where hypergamy was, in fact, substantially less prevalent than hypogamy. Therefore, I advise caution in attributing contextual influences solely to the normative climate. Instead, it should be treated as one of the representative factors, explaining some, but not all, variations in the well-being outcomes of educational sorting across countries and cohorts. Second, this study did not explore variations in the outcomes of assortative mating across the educational ladder. For instance, partnering downwards may carry different meanings between women with tertiary education and those without \parencite{potarcaAreWomenHypogamous2022}. This is largely a methodological limitation, since the DMM averages the effects across all individuals, masking potential variations \parencite{sobelDiagonalMobilityModels1981,sobelSocialMobilityFertility1985}. Third, the cross-sectional data on hand prevented this study from thoroughly addressing endogeneity issues arising from individuals' self-selection into or out of certain relationship types. Though findings remain robust in subpopulations such as married respondents and those never divorced, I caution against drawing causal conclusions without longitudinal data.

These findings and limitations highlight new directions for future research on educational sorting and subjective well-being. Future studies may consider differentiating the outcomes by educational level, utilizing longitudinal data to address selection issues, and investigating other contextual factors relevant to educational sorting patterns. Additionally, future work may broaden the scope of status comparison between partners, encompassing other sociodemographic characteristics important to mate selection and contexts beyond Europe.
