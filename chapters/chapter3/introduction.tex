\section{Introduction}
\label{sec:ch3-introduction}

Since \citeauthor{easterlinDoesEconomicGrowth1974}'s \citeyear{easterlinDoesEconomicGrowth1974} pioneering work, scholars have increasingly studied inequalities in subjective well-being within and across nations, offering insights, beyond those provided by objective or external measures, into individuals' subjective experience and evaluation of life, and more broadly, into the often-hidden inequity between social groups living close to their aspirations and those who are not \parencite{alesinaInequalityHappinessAre2004,dienerNationalDifferencesReported1995,dienerSubjectiveWellBeingScience2009}. These well-being disparities have been examined in light of various factors reproducing social stratification, including income distribution \parencite{hochmanImpactWealthSubjective2013,ngamabaIncomeInequalitySubjective2018}, intergenerational mobility \parencite{schuckDoesIntergenerationalEducational2018,wangEducationalMobilitySubjective2024}, and mate selection \parencite{potarcaAreWomenHypogamous2022,zhaoPartnersEducationalPairings2022}. This study focuses on the well-being implications of educational sorting in unions, set against the changing landscape and vast diversity of educational pairings across Europe \parencite{eratEducationalAssortativeMating2021,esteveEducationalHomogamyGap2013,esteveEndHypergamyGlobal2016}.

As well-documented by previous studies, educational sorting in relationships has far-reaching implications for individuals' subjective well-being. This study first revisits three pre-established propositions: the homogamy advantage, returns to status attainment, and gender roles. Homogamy, where partners share the same level of education, is often considered the optimal relationship configuration, as their compatibility in socioeconomic resources and other intangible attributes, such as attitudes, values, and knowledge structures, is conducive to relationship stability and satisfaction \parencite{gauntCoupleSimilarityMarital2006,kalmijnIntermarriageHomogamyCauses1998,keizerAreEqualsHappier2015}. Others, to whom partnership is a channel for status attainment and upward mobility, may experience positive well-being returns when partnering upwards in education \parencite{schwartzMarryingMarryingStatus2016,zhaoPartnersEducationalPairings2022}. Still, gender roles cannot be overlooked \parencite{centersConjugalPowerStructure1971,westDoingGender1987}. Educational m matches have a profound influence on power dynamics within relationships \parencite{eeckhautEducationalHeterogamyDivision2014}, and the (dis)satisfaction arising from these mismatches has much to do with whether the relationship adheres to aspired gender roles, which, often differ between genders and across societies \parencite{knightOneEgalitarianismSeveral2017,vanbavelReversalGenderGap2018}.

Apart from testing these propositions, this study extends the previous literature by addressing whether and how individuals, as social beings, experience the well-being implications of educational sorting differently across contexts. Europe, with its rich tapestry of cultures, economies, and politics and its fast-changing, yet highly diverse, demographic landscape, provides an exemplary field \parencite{artsModelsWelfareState2010,dehauwReversedGenderGap2017,eratEducationalAssortativeMating2021,esping-andersenThreeWorldsWelfare1990,esping-andersenWelfareRegimesSocial2015}. From long-established democracies to post-socialist states, from capitalist economies to social democratic regimes, and from individualistic to familial cultures, these diversities are associated with variations in the patterns of educational pairings. These diversities include, for instance, the sustained dominance of homogamy across cohorts, yet to a lesser extent in the Nordics compared to countries with strong familism in Southern Europe and post-socialist states in Central \& Eastern Europe \parencite{domanskiEducationalHomogamy222007}, as well as the rising hypogamy o taking hypergamy, but less so in Continental Europe \parencite{dehauwReversedGenderGap2017,eratEducationalAssortativeMating2021}. Of primary concern to this study, these contextual differences may lead to diverging well-being outcomes of educational sorting. I expect that homogamy, for the benefit of status maintenance, is more advantageous in regimes with welfare designed to reinforce, rather than reduce, inequality; partnering upwards in education yields greater well-being returns in contexts where it is crucial for upward mobility; and gender roles, manifested, for instance, in the relative advantage of male-dominated relationships (i.e., female hypergamy) for men, are more influential in societies with traditional gender values. To explore these variations, I adopt a clustering approach with reference to welfare typologies, educational and socioeconomic conditions, and demographic patterns, as detailed in later sections.

To further develop the contextual dimension, I go beyond geographical boundaries to investigate whether and how the well-being outcomes of educational sorting are conditioned by the normative climate varying across time and space. Two critical aspects—the dominance of homogamy over heterogamy, and the dynamics between hypergamy and hypogamy—are of particular interest. According to the social disapproval hypothesis, I expect certain types of educational sorting to yield greater benefits, or incur fewer penalties, if they are more common in a particular context \parencite{kalmijnCountryDifferencesEffects2010,soonsMarriageMoreCohabitation2009}. Alternatively, from a selection perspective, those making non-conformist decisions, such as hypogamous couples in hypergamy-dominated contexts, may be driven by stronger personal conviction, potentially resulting in more positive well-being outcomes \parencite{potarcaAreWomenHypogamous2022}.

Methodologically, unlike previous studies, I estimate the influence of educational sorting per se—the difference in education between partners—independent of each partner's education. The linear dependency among these factors has led most studies to model the effects of educational sorting while controlling for only one partner's education, resulting in estimations confounded by the omitted factor \parencite{eeckhautAnalysingEffectEducational2013}. For instance, the benefits of hypergamy may be overestimated, and the potential distress it causes women may be overlooked if the male partner's higher education, which is often linked with better well-being, is omitted. This methodological issue is particularly concerning in cross-national studies, as the well-being returns to education differ across contexts, which, if not controlled for, may further distort the understanding of the contextual variations in the well-being implications of educational sorting. This study contributes to the emerging, yet still scarce, literature that employs the Diagonal Mobility Model (DMM) for more precise estimations, as detailed in later sections \parencite{sobelDiagonalMobilityModels1981,zangMobilityEffectsHypothesis2023}.

Based on these arguments, this study addresses two key questions: How does educational sorting impact individuals' subjective well-being, independent of their own and their partner's education? How do these associations differ between genders and vary across societies and normative climates in Europe? To answer these questions, the study uses the DMM to analyze a sample of 180,733 respondents living with a heterosexual partner in marriage or cohabitation from 29 European countries, selected from Rounds 1-10 (2002-2020) of the European Social Survey (ESS).

This study is among the few to add a contextual dimension to the well-being outcomes of educational sorting, thereby bridging the literature on changing educational sorting patterns with the scholarship on well-being disparities across countries. It provides unique insights into how individuals' well-being is contingent upon relationship dynamics and underscores the importance of considering the sociocultural, economic, and demographic backgrounds that shape these outcomes.

The article is organized as follows. I first revisit the hypotheses raised to explain the well-being outcomes of educational sorting. I then examine the contextual variations across societies and normative climates. This is followed by an introduction of data, measures, and analytical strategies. Subsequently, I present sample characteristics, educational sorting patterns, and empirical findings. I conclude by summarizing the results and discussing their limitations and implications for future research.
