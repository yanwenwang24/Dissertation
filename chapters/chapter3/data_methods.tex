\section{Data and Methods}
\label{sec:ch3-data-methods}

\subsection{Data and Sample}

Samples were selected from Rounds 1-10 (2002-2020) of the European Social Survey (ESS). The ESS is a pan-European biennial survey with randomly chosen respondents representative of the non-institutionalized population aged 15 and above in each participating country, covering an extensive range of data including attitudes, wellbeing, and sociodemographic characteristics. For its consistent and comparable design across countries and periods, the ESS is the ideal data source for studying the well-being implications of educational sorting across contexts.

I performed stepwise sample restrictions, as detailed in Table~\ref{app:tab:sample_restriction_ch3}. First, I pooled respondent from Rounds 1-10 of the ESS and selected those in marriage or cohabitation with a heterosexual partner residing in the same household, reducing the sample size from 480,625 to 271,522 by 43.51\%. Both married and cohabiting respondents were included to account for the increasing prevalence of cohabitation and the varying selectivity of marriage across countries. Second, I selected respondents between ages 25 and 65 to minimize potential selection bias related to further educational advancement, early union formation, and bereavement, further reducing the sample size to 213,303 by 21.44\%. Next, I applied list-wise deletion of respondents with missing values for key variables, reducing another 8.34\%. Finally, I dropped respondents from countries with fewer than three rounds of data, as well as those from Israel, a non-European country, dropping an additional 7.56\%. The decision to exclude countries with limited rounds was made to ensure reliable estimation of region-specific patterns, as their inclusion could distort regional analyses when surveys were conducted only in very early or recent periods. Having at least three time points also allows for more robust measures of country-period-specific educational sorting climates and their moderating effects \parencite{eratEducationalAssortativeMating2021}. The final analytic sample includes 180,733 respondents from 29 European countries, as listed in Table~\ref{app:tab:country_clusters}.

Following the ESS's guidelines, I applied analysis weights—the product of post-stratification and population weights—to adjust for errors of sampling and non-response, as well as variations in inclusion probabilities and population size.

\subsection{Individual Measures}

\textbf{Subjective Well-Being.} The dependent variable, subjective well-being, is measured by a single-item question assessing respondents' life satisfaction on a scale from 0 (\textit{extremely dissatisfied}) to 10 (\textit{extremely satisfied}): “All things considered, how satisfied are you with your life as a whole?” As shown by previous studies, this survey item has been widely used for measuring life satisfaction and is a reliable single-item indicator of subjective well-being across contexts \parencite{cheungAssessingValiditySingleitem2014,dienerSubjectiveWellBeingScience2009}.

\textbf{Educational Difference.} To measure educational difference between partners, I first categorized educational attainment measured by the International Standard Classification of Education (ISCED) into four levels: lower secondary or less (ISCED \RNum{1} \& \RNum{2}), upper secondary (ISCED \RNum{3}), non-tertiary (ISCED \RNum{4}), and tertiary (ISCED \RNum{5} \& \RNum{6}). This ensures sufficient sample sizes at each level while preserving as much distinction as possible across the educational spectrum. I then constructed a dummy variable denoting whether the respondent's education matches their partner's (homogamy vs. heterogamy), and a trichotomous indicator denoting homogamy, hypergamy if the male has higher education than the female, and hypogamy if otherwise.

\textbf{Covariates.} Several covariates that may confound the association between educational sorting and subjective well-being were controlled for. These include demographic information such as gender (1=\textit{female}, 0=\textit{male}), age (standardized), cohabitation (1=\textit{cohabiting}, 0=\textit{married}), ever-divorced (1=\textit{ever-divorced}, 0=\textit{no divorce}), immigration (1=\textit{immigrant}, 0=\textit{non-immigrant}), ethnicity (1=\textit{minority}, 0=\textit{majority}), household size (a continuous variable capped at 6), number of children (a continuous variable capped at 3), and the presence of children under age 6 (1=\textit{having at least one child under age 6 living in the same household}, 0=\textit{otherwise}), as well as socioeconomic characteristics indicated by unemployment (1=\textit{unemployed during the last 7 days}, 0=\textit{otherwise}) and the subjective feeling of household income, a dichotomous variable denoting whether one is \textit{living comfortably on present income}. Dummies for countries and periods were also included to account for spatial and temporal variations.

\subsection{Contextual Measures}

\textbf{Country Cluster.} As previously discussed, the association between educational sorting and subjective well-being may vary across contexts. I categorized the 29 countries under investigation into six clusters: Anglo-Saxon countries, the Baltic States, Central \& Eastern Europe, Continental Europe, Nordic countries, and Southern Europe, as listed in Table~\ref{app:tab:country_clusters}.

\textbf{The Normative Climate.} For consistency and comparability with earlier studies \parencite{eratEducationalAssortativeMating2021,esteveGenderGapReversalEducation2012}, I used the H-index to measure two aspects of the normative climate: the dominance of homogamy over heterogamy, and the dynamics between hypergamy and hypogamy. The H-index is given by:

\begin{equation}
    \label{eq:h-index}
    H = \ln\left(\frac{A}{B}\right)
\end{equation}

where, in terms of the dominance of homogamy, $A$ represents the number of homogamous unions and $B$ the number of heterogamous ones, and with regard to the dynamics between hypergamy and hypogamy, $A$ denotes the number of hypergamous unions and $B$ the number of hypogamous ones. Both indexes are positive if homogamous or hypergamous unions outnumber their respective counterparts. To incorporate both spatial and temporal variations, the H-indexes were calculated per country and period.

\subsection{Analytical Procedures}

In the first step, I employed the DMMs on respondents pooled from all countries. Since the attributes sought in a partner, as well as the aspired gender roles in relationships, are often asymmetric between men and women, I stratified the samples by gender and analyzed them separately. This approach is preferred over models with gender interaction terms, since it considers potential gender variations in the wellbeing implications of covariates, such as household income and the presence of children, as well as the gender gaps in well-being across contexts. These DMMs include the baseline model that estimates the relative influence of respondents' and their partner's education, while controlling for all covariates, and subsequent models that estimate the effects of educational differences between partners with dummies for heterogamy, hypergamy, and hypogamy. Gender differences were assessed using Z-statistic $(\beta_{m} + \beta_{f})/\sqrt{\text{SE}_{m}^{2} + \text{SE}_{f}^{2}}$, where $\beta$ and $\text{SE}$ denote the coefficients and standard errors of educational sorting indicators, respectively.

Next, to explore contextual variations across societies and normative climates, I first repeated the above models on samples further stratified by country cluster. I then conducted moderation analysis on pooled samples, with interaction terms between educational sorting indicators and country-period-specific H-indexes of homogamy and hypergamy.

I used the Akaike Information Criterion (AIC) and likelihood ratio tests (LRT) to assess if adding any new variables would significantly improve the model fits \parencite{sobelSocialMobilityFertility1985}. All models were estimated using the "gnm" package in R \parencite{turnerGnmGeneralizedNonlinear2005}.

\subsection{Auxiliary Analyses}

Although the cross-sectional data at hand prevented me from ruling out the underlying selectivity of educational sorting, I conducted auxiliary analyses to check if the results hold for various subpopulations of the sample. First, I repeated the analyses on a subsample consisting of only married respondents, as those in cohabitation, as shown by previous studies, were more prone to lower well-being \parencite{soonsMarriageMoreCohabitation2009}, held less traditional gender values \parencite{dominguez-folguerasCohabitationMoreEgalitarian2013}, and had different patterns of mate selection \parencite{esteveEducationalHomogamyGap2013}. Therefore, conducting analyses specifically on married respondents helps preclude the selectivity of cohabiting partners with lower subjective well-being into certain types of relationships.

Next, I conducted analyses on a subsample of married or cohabiting respondents who had never been divorced, to address the selection out of marriage by those with lower subjective well-being, who may self-select into certain relationship configurations (e.g., heterogamy) known to have higher divorce risks \parencite{kalmijnUnionDisruptionNetherlands2003,kalmijnCountryDifferencesEffects2010,theunisHisHerEducation2018}. Since over 80\% were either married or without a divorce history, instead of comparative analyses, I focused on these subpopulations only.

Finally, given that previous research has demonstrated that measures of educational sorting patterns and potentially, the well-being outcomes of educational sorting, are sensitive to how education is categorized (e.g., \cite{gihlebEducationalHomogamyAssortative2020}), I conducted additional analyses using both three-level (\textit{lower secondary or less}, \textit{upper secondary}, and \textit{non-tertiary or above}) and five-level classifications (\textit{less than lower secondary}, \textit{lower secondary}, \textit{upper secondary}, \textit{non-tertiary}, and \textit{tertiary or above}). The five-level categorization is the finest distinction possible given the harmonized educational data across European countries in the ESS, especially in early rounds. These analyses serve as robustness checks to ensure that the relationship between educational sorting and well-being is not an artifact of selective educational categorization.
