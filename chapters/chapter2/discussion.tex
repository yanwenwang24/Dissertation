\section{Discussion}
\label{sec:ch2-discussion}

The current study utilized the DMM to explore the well-being implications of intergenerational educational mobility for primary movers and their parents, with a focus on those from only-child families. The results indicate a concurrent influence of parents' and adult children's education on their subjective well-being, with individuals' own education being more influential. Notably, I observed distinct and asymmetric effects of mobility exclusively among only-child families: both generations experienced decreased life satisfaction in the event of downward mobility, while upward mobility did not have any significant impact. Furthermore, I did not find evidence supporting gender differentiation or traditional son preferences within only-child families.

Theoretically, I embraced the linked lives perspective and extended mobility theories to include the experiences of primary movers' parents. In the Chinese context, where education mobility is pursued as a family project, and parents harbor high aspirations for their children's upward mobility, along with deep-seated anxiety of falling downward \parencite{guSacrificeIndebtednessIntergenerational2022,guWhyChineseAdolescent2021,mengWhenAnxiousMothers2020}, I have strong reasons to expect parents' subjective well-being to be closely linked to their children's educational attainment and mobility. Furthermore, I underscore the saliency of family structures and gender dynamics in shaping well-being outcomes of mobility. In the now dominating only-child families, the child, regardless of gender, is the only hope for the family to move upward. While the progress toward gender equality owes much to the rise of only-child families, few studies have made comparisons within only-child families \parencite{fongChinasOneChildPolicy2002,liuWhoBenefitsBeing2021,wuFertilityDeclineWomens2014}. In this study, I probe whether the well-being ramifications of educational mobility are amplified for only children and whether the son- and daughter-parents relationships within only-child families have converged based on the experiences of both generations.

Empirically, the results affirmed the relevance of the acculturation theory for both primary movers and their parents in only-child families. Both generations' subjective well-being was simultaneously impacted by educational statuses at origin and destination, with individuals' own status being more influential. These findings echo previous studies on the impact of mobility on primary movers' well-being both within China and globally \parencite{dhooreSocialMobilityLife2019,kwonImpactIntergenerationalMobility2022,schuckDoesIntergenerationalEducational2018,zangFrustratedAchieversSatisfied2016}, and are in line with the extant literature on the intergenerational associations between children's education and parents' well-being \parencite{chenEducationFeverChina2021,maDoesAdultChildrens2022}.

Empirical results show that in only-child families, both generations' subjective well-being was negatively affected by downward mobility without any reward from moving upward. This was, however, not observed among those with siblings, nor by previous studies on the general Chinese population, though their data and measures of intergenerational mobility were different \parencite{zangFrustratedAchieversSatisfied2016,zhaoDifferentialAcculturationStudy2019,zhaoInterIntragenerationalSocial2017}. The unique intergenerational dynamics in only-child families, where the entire family's hope for moving upward rests on the only child, intensify the negative consequences associated with not achieving this goal. With these findings, I contend that the consequences of intergenerational mobility should be contextualized in families and examined with consideration of diverse familial settings.

Two reasons may explain why parents and children from only-child families were adversely impacted by downward mobility without benefiting from upward mobility. First, from a cognitive and psychological perspective, individuals tend to be more sensitive to the pains associated with losses—downward mobility in this case, than to the pleasures derived from equivalent gains \parencite{kahnemanAnomaliesEndowmentEffect1991}. Second, due to social comparison, individuals’ subjective well-being is influenced less by absolute gains or losses and more by their relative standing compared to others. When most individuals, especially those from only-child families, have moved upward during mass educational expansion in China, the gains from upward mobility are likely to diminish, and the losses associated with downward mobility become even more devastating.

These well-being consequences of educational mobility did not differ by gender among primary movers. Sons and daughters, with or without siblings, experienced similar outcomes. For parents from only-child families, the education and mobility of the daughter mattered as much as those of the son to fathers, and even more so to mothers. Previous studies have attributed the overall improvement in women's educational attainment in part to the rise of only-child families \parencite{fongChinasOneChildPolicy2002,wuFertilityDeclineWomens2014}. This study furthers the literature by contrasting families with only sons against those with only daughters and by exploring the dynamics between different parent-child gender pairs. Indeed, traditional son preferences have been undermined by only-child status, and the once heavily gendered son- and daughter-parent relationships have converged and, to some extent, reversed in favor of the daughter \parencite{guWhyChineseAdolescent2021}.

The study is, however, not without limitations. First, despite the strengthening mother-child status association \parencite{huGenderEducationExpansion2023}, measures of intergenerational mobility did not differentiate between the genders of parents. Incorporating both parents' education separately will add another dimension to the DMM's diagonal designs, which complicates the model and may violate its central assumption—the representation of each non-diagonal level by non-mobile individuals. Second, the same methodological concern prevented me from including multiple children's education statuses, thereby limiting the findings of parents to those from only-child families. Third, I did not investigate variations in the outcomes of mobility across the educational ladder, since the DMM averages the effects across all individuals. Fourth, the cross-sectional design of the DMM constrains my ability to thoroughly address endogeneity issues arising from individuals' selection into different mobility trajectories. Therefore, I advise caution in drawing causal inferences from the findings without the support of longitudinal data. Lastly, my analysis relies on the CPFS data collected in 2010. Considering the changes within the educational system and broader society, this temporal gap may affect the generalizability of the results to the present context.

These limitations did not diminish the strength of this study. While mobility research has predominantly focused on primary movers, I demonstrated the importance of embracing the perspective of linked lives in contextualizing educational mobility within intergenerational relationships and recognizing its nature as a family project. Findings showed that educational mobility mattered not only to primary movers but also to their parents, with outcomes conditioned by the only-child family structure and parent-child gender dynamics. These insights, along with identified limitations, pave new avenue for future research, including but not limited to employing gender-sensitive mobility measures, addressing family dynamics with multiple children with attention to gender and birth order, and exploring heterogeneous mobility effects across social strata. Future research may also leverage more recent data to investigate the complex interplay between individual mobility and broader structural changes. This may involve developing contextual-level indexes to capture the degree of competitiveness for moving upward or measuring relative educational mobility by comparing the relative educational positions of parents and children within their respective temporal and spatial groups, both of which, however, have not been accomplished by the current study. Moreover, I call for future studies to incorporate other dimensions of intergenerational mobility, such as occupational and income mobility, and to assess the universality or specificity of these findings across different contexts, to better understand the full spectrum of social stratification and its implications for well-being.
