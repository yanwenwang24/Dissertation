\section{Theoretical Framework}
\label{sec:ch2-theoretical-framework}

\subsection{Educational Mobility and Subjective Well-Being}

Theories on the subjective well-being outcomes of mobility, albeit diverse in perspectives and propositions, converge on two key aspects: their focus on primary movers and the conceptualization of intergenerational mobility in terms to three essential components—parents' status (origin), individuals' own status (destination), and net mobility. Each component may have independent effects, and jointly, they deem mobility consequential to primary movers' subjective well-being.

\citeauthor{blauSocialMobilityInterpersonal1956}'s (\citeyear{blauSocialMobilityInterpersonal1956})
acculturation theory emphasizes the impact of origin and destination statuses on primary movers' subjective well-being. As individuals ascend or descend the social ladder, they may find themselves neither fully acculturated to the value and lifestyles of the destination class nor entirely constrained by the origin. At an intermediate position, their normative beliefs, behaviors, and well-being are under the concurrent influence of origin and destination. In the context of educational mobility, \citeauthor{blauSocialMobilityInterpersonal1956}'s theory suggests that the educational statuses of both individuals and their parents are likely strong predictors of well-being. However, while the relative influence of the two statuses—the degree of influence by one compared to the other—may reflect the intermediate positions of primary movers, it does not account for the influence of mobility \textit{per se}, that is, the difference between origin and destination.

Other theories, while acknowledging the influence of educational statuses, underscore the independent effects of mobility. The Multiple Discrepancies Theory (MDT) posits that individuals' subjective wellbeing is largely determined by the perceived discrepancies between expectations and realities in various life domains, including education \parencite{michalosMultipleDiscrepanciesTheory1985}. Accordingly, upward mobility, which aligns with personal, familial, and societal expectations, tends to enhance life satisfaction. In contrast, those experiencing downward mobility or stagnation may feel relatively deprived. \citeauthor{newmanFallingGraceDownward1999}'s (\citeyear{newmanFallingGraceDownward1999}) anthropological work delved into the lives of individuals in downward mobility in the United States and noted that the age of affluence had exacerbated their tormented experience of "falling from grace." When upward mobility has become an assumed given, as is the case in China during mass educational expansion \parencite{xieTrendsSocialMobility2022,yeungHigherEducationExpansion2013}, the detrimental effects of falling downward are likely amplified, becoming more substantial than the benefits of upward mobility.

Still, mobility in either direction can be disruptive. \citeauthor{sorokin1927social}'s (\citeyear{sorokin1927social}) dissociative theory points to the psychological toll experienced by individuals transitioning between positions, often feeling estranged from both. It was concurred by \citeauthor{straussContextsSocialMobility1971}'s (\citeyear{straussContextsSocialMobility1971}) qualitative interviews with upwardly mobile individuals, who expressed a sense of discontinuity with their enduring identities and disappointment on arrival in search for a yet higher position. In opposition, \textcite{goldthorpeSocialMobilityClass1980} found little support for the dissociative theory, probably because upwardly mobile individuals in large numbers formed their own groups, established extensive social contacts, and thereby avoided marginalization.

These theories concertedly suggest that the three components of intergenerational educational mobility—individuals' education, parents' education, and net mobility—are all potential sources of influence. Distinguishing them apart is meaningful and crucial for understanding faithfully the well-being consequences of educational mobility. While early quantitative studies lacked the tools to disentangle their linear dependency, recent studies using the DMM have yielded mixed evidence across contexts. For example, \textcite{schuckDoesIntergenerationalEducational2018} found that in most parts of Europe, individuals' subjective well-being was influenced simultaneously by their own and their parents' education, with the former exerting a greater relative influence. Only in Continental Europe, where education was most salient for reproducing social stratification, were independent mobility effects observed: positive for those moving upward and negative for those on downward trajectories. In China, mobility studies, though not specifically on educational mobility, estimated that the relative influence of individuals' and parents' status on primary movers' well-being was approximately 3 to 1, and did not find any independent mobility effects once statuses have been accounted for \parencite{zangFrustratedAchieversSatisfied2016,zhaoDifferentialAcculturationStudy2019,zhaoInterIntragenerationalSocial2017}.

To date, theories and empirical studies have centered on primary movers and fallen short of contextualizing the experience of mobility in intergenerational relationships. This oversight is largely due to the insufficient recognition of intergenerational mobility as a family project, in which family members, especially parents, are extensively involved. In the next section, I embrace the linked lives perspective and discuss the first contribution of this study, that is, the extension of the theoretical frameworks to encompass the older generation.

\subsection{Parents at the Other End of Educational Mobility}

The idea of linked lives—"the embeddedness of human lives in social relationships of kin and friends that extend across the life span"—is key to understanding the intergenerational mutual influence between parents and children throughout the life course \parencite{elderFamiliesLivesDevelopments1987,macmillanFamiliesLifeCourse2005}. While a growing body of literature, from the perspective of linked lives, has demonstrated that parents' well-being is strongly influenced by life events and conditions of children, including health, marriage, education, and employment (e.g., \cite{friedmanSchoolingOffspringSurvival2014,maDoesAdultChildrens2022,zhangChildrensTransitionsAdulthood2023}), research on the c sequences of mobility is largely limited to primary movers. This study argues that educational mobility is a family project in which both parents and children are highly involved, thus the well-being implications of educational mobility likely resonate across generations. Specifically, I posit that the extant theoretical framework—mobility as a concurrent influence of origin, destination, and net mobility—is suitable for explaining parents' subjective experience of children's educational mobility.

First, parents' subjective well-being is strongly associated with not only their own education but also their children's \parencite{chenEducationFeverChina2021,maDoesAdultChildrens2022}. Studies on intergenerational reciprocity have shown that parents of more educated children receive from them more resources and fewer demands, which leads to lower mortality risks and better mental health \parencite{friedmanSchoolingOffspringSurvival2014,maDoesAdultChildrens2022}. In China, without the need to trace down the cause, parents' extensive, sometimes frenetic involvement in children's education is itself the best illustration of how much children's education means for parents' well-being. They participate extensively in their children's homework and school activities, spend large sums on extracurricular courses, and even purchase expensive apartments near elite schools for admission \parencite{fengSchoolQualityHousing2013,zhouFamilySocioeconomicEffect2015}. \textcite{chenEducationFeverChina2021} found that even for school-aged children, long before any material returns, their class rankings had already significantly affected parents' life satisfaction, particularly among urban, middle-class, and only-child families. \textcite{tongChildrensAcademicPerformance2021} found that parents' subjective well-being was strongly associated with both the learning efforts and academic performance of their children.

Second, mobility should be distinguished apart, with a potentially standalone effect on parents' subjective well-being. In China, parents harbor high aspirations not only for their children's educational achievement but also for upward mobility \parencite{duChinesePerceiveUpward2021}. \textcite{guSacrificeIndebtednessIntergenerational2022} pointed out that children's education, pursued as a family project, was deeply rooted in families' social mobility aspirations. \textcite{xuHopingPhoenixShanghai2013} interviewed Shanghai fathers with daughters and reported that the fathers, regardless of their own education level and despite involving less than mothers, strongly expected their daughters to exceed themselves, and regarded the fulfillment of such aspirations as their most important responsibilities.

Accompanying the highly esteemed upward mobility are the desire for status maintenance and the avoidance of downward mobility, both of which are key motives behind families' educational decisions \parencite{breenExplainingEducationalDifferentials1997,mengWhenAnxiousMothers2020}. This is especially relevant in the Chinese context where the pace of educational expansion has slowed down, the once taken-for-granted upward mobility has been hindered \parencite{yeungHigherEducationExpansion2013}. Several scholars have noted the pervasive and deep-seated fear of falling downward, which compels parents to secure their social status by every possible means, including strengthening their children's educational capital \parencite{ehrenreichFearFallingInner1989,mengWhenAnxiousMothers2020}. Thus, downward mobility is often considered the worst scenario, with its detrimental impact outweighing the potential benefits of upward mobility.

In extending mobility theories to include parents, I am sensitive toward the potential differences between fathers' and mothers' subjective experience of children's educational mobility. Though both parents aspire for their children's educational achievement and upward mobility, mothers often engage more intensively in children's educational activities, adhering to child-centered, time-consuming, self-sacrificing “intensive mothering” practices \parencite{mengWhenAnxiousMothers2020,muChangingPatternsDeterminants2022}. Considering these gender differences in parenting intensity, I expect children's education and mobility to have a more pronounced impact on mothers' subjective well-being compared to fathers.

\subsection{Gender Dynamics and Only-Child Families}

The second contribution of this study, in extending the theoretical framework to include parents, is the investigation into the roles of family structures in shaping the well-being outcomes of intergenerational educational mobility. Here, I focus on one of the most striking demographic shifts in China during the past few decades—the rise of only-child families. By the end of the One-Child Policy in 2015, only-child families have accounted for 65.6\% of all families with children. With sustained sub-replacement fertility rates, the percentage has continued to rise.

My first step is to compare only children with those who have siblings. Previous literature on the unique intergenerational dynamics in only-child families has laid a solid foundation for theorization \parencite{falboQuantitativeReviewOnly1986}. Being an only child often means receiving undivided parental attention and resources, leading to more responsive, intimate, and higher-quality parent-child relationships \parencite{liuWhoBenefitsBeing2021}. Only children often outperform others in academic performance and educational attainment \parencite{falboQuantitativeReviewOnly1986}. These advantages may translate to only children's unique subjective experience of educational mobility in two ways. First, the lack of resource dilution allows them to be more dependent on their parents \parencite{liuWhoBenefitsBeing2021}. Parents' education, in relation to their own, may be of greater relative influence compared to children with siblings. Second, the privileges associated with only-child status may come at a cost. Being the family's sole hope for upward mobility, only children could not afford to fail parental expectations and thus may experience heightened ramifications of mobility, particularly in the worst scenario of falling downward.

My second step, informed by the literature on gender equality and fertility decline, is to examine whether gender dynamics within only-child families affect the subjective experience of mobility. The rise of only-child families, in which the child, regardless of gender, receives full educational support from parents, has significantly eroded traditional son preferences that dominated much of China's history. Using the One-Child Policy as an exogenous factor, \textcite{wuFertilityDeclineWomens2014} demonstrated that women's empowerment, rather than being the cause of fertility decline, was a result of the increasing prevalence of only-child families. Daughters have benefited more than sons in terms of years of schooling and subsequent occupational attainment. These phenomena have led some scholars to argue that the once heavily gendered son- and daughter-parent contracts have converged at the very least, if not in favor of daughters \parencite{fongChinasOneChildPolicy2002,guSacrificeIndebtednessIntergenerational2022,xuHopingPhoenixShanghai2013}.

However, the only-child status may not have thoroughly undermined gender stereotypic expectations. \textcite{liuBoysOnlychildrenGirls2006} interviewed a sample of parents from only-child families and reported their persistent reinforcement of gender stereotypes in educational expectation. Therefore, she suggested that without an empirical comparison between only-son and only-daughter families, it would be overly optimistic to draw from women's overall improved educational attainment the complete eradication of son preferences by the only-child status.

In addition to the comparison by the only child's gender, I further consider gender dynamics by parent-child pairwise dyad—mother-daughter, mother-son, father-daughter, father-son dyads in only-child families. \textcite{huUnmakingOccupationalGender2024}, in their theorization of the intergenerational transmission of gender ideology, distinguished homo-lineal gender-role learning from hetero-lineal gender boundary-setting, in which the same-sex parent passes on his or her gender-role attitudes to their children, and the opposite-sex parent erects gender boundaries. Their theories, in the Chinese context, apply only to daughters but not to sons, reflecting the variations not only by fathers' and mothers' distinct influence but also by the gender of the child. Furthermore, a number of studies have noted the asymmetric endorsement of traditional son preferences between fathers and mothers, as the presence of sons may lead parents to endorse more conservative gender values, yet with more pronounced influence on mothers than fathers \parencite{sunAreMothersSons2017}. Thus, though without conclusive evidence, I infer from previous studies that traditional son preferences, if having persisted within only-child families, may be endorsed by mothers to a greater extent than by fathers.

Taking stock of the literature, the current study highlights the need for investigating parental gender preferences within the context of only-child families, with consideration of potential differences between father and mothers. Specifically, I look into the gender dynamics embedded in the well-being implications of intergenerational educational mobility within only-child families. Had the traditional son preferences prevailed would only sons assume greater responsibility for achieving higher educational attainment. This translates to, first, gender differences in only children's well-being outcomes of mobility, and second, variations in parents' well-being outcomes based on the child's gender, with potential differences between fathers and mothers. I examine these gender-related differences with regard to the relative influence of the child's education (compared to parental education), as well as the effects of net mobility.
