\section{Results}
\label{sec:ch2-results}

\subsection{Descriptive Statistics}

Table~\ref{tab:descriptive_stats} summarizes the descriptive statistics for the samples. Gender-specific statistics for those from only-child families are detailed in Table~\ref{app:tab:descriptive_by_gender}. Among parents from only-child families (N = 3808), gender composition was nearly even, with 47.98\% being fathers. Sons (59.46\%) were overrepresented among only children (N = 1475), compared to 47.21\% among those with siblings. The gender imbalance among only children can be attributed to two factors. First, the historical One-Child Policy has intensified sex selection at birth in favor of sons \parencite{zhuChinasExcessMales2009}. Second, only sons are more likely than only daughters to co-reside with their parents and to be included in household-level surveys \parencite{fengChengshiDushengZinu2009}.

Parents from only-child families reported higher life satisfaction ($M$=3.43, \\$SD$=1.05) than adult children with or without siblings. They were on average 54.10 years old ($SD$=7.88), with fathers older than mothers. Over 90\% were married, and 95.69\% identified themselves as Han Chinese. Approximately one in ten were members of the Communist Party, and around two thirds (67.88\%) lived in urban areas. Most of these parents (73.82\%) considered themselves economically related to their only child.

    Among the younger generation, children with or without siblings reported similar life satisfaction. Daughters were more satisfied with life than sons. The average age of only children was 30.44 years ($SD$=7.82), significantly younger than those with siblings. Correspondingly, a smaller proportion of only children were married (64.27\%), compared to those not restricted by only-child status (85.88\%). Over 90\% were Han Chinese. Only 5.59\% were Communist Party members. Urban residency and economic relationships with parents were more common among only children than those with siblings.

Unsurprisingly, the younger generation had much higher educational attainments than their parents. Among only children, 32.54\% held college or higher degrees, 21.90\% had high school diplomas, 27.12\% had completed middle school, and only 18.44\% had primary school or lower education. In contrast, in the other two groups, 75.02\% of parents from only-child families and 74.91\% of primary movers unrestricted by only-child status did not progress beyond middle school. I found evidence consistent with the reversal of gender gaps in education, particularly among only-child families \parencite{yeungHigherEducationExpansion2013}. As shown in Table~\ref{app:tab:descriptive_by_gender}, while fathers were typically more educated than mothers, daughters have now surpassed sons in educational attainment.

Table~\ref{tab:mobility_patterns} outlines the patterns of intergenerational educational mobility among those from only-child families and all primary movers with or without siblings. Upward mobility was notably prevalent, especially among only-child families. Approximately 56\% of only children moved upward, compared to 41.67\% of those with siblings. A small but non-negligible percentage (around 10\%) of individuals in both groups experienced downward mobility. Additionally, a considerable proportion of individuals remained non-mobile at each educational level, totaling 34.14\% in only-child families and 46.87\% among those with siblings. Those non-mobile individuals are essential in the DMMs for representing their respective positions and serving as reference groups \parencite{sobelSocialMobilityFertility1985}.

\subsection{Findings}

In the first step, I replicated previous research on the subjective experience of educational mobility among primary movers. Table~\ref{tab:dmm_results} presents the results of the DMMs on life satisfaction of primary movers with or without siblings, organized in four parts: the relative influence of parents' (origin) and individuals' (destination) education, mobility indicators, diagonal intercepts of non-mobile individuals, and covariates.

The baseline model estimates the relative influence of origin and destination. Results show that the subjective well-being of primary movers was affected by both their parents' (0.29) and their own (0.71) education, with the latter of greater influence. In Model 2, I added dummy variables for upward and downward mobility. Neither was significant. This suggests that net of educational statuses at the origin and destination, mobility in either direction did not independently affect primary movers' subjective well-being. These results support the acculturation hypothesis (H1) and rejected the mobility hypotheses (H2, H2a, and H2b), with regard to primary movers. These findings are consistent with findings of other studies in China, in which individuals' own status accounted for a substantial proportion (ranging from 76\% to 79\%) of the total status effects at origin and destination, and no mobility effects were observed after accounting for statuses \parencite{zangFrustratedAchieversSatisfied2016,zhaoDifferentialAcculturationStudy2019,zhaoInterIntragenerationalSocial2017}.

In Models 3 and 4, I interacted only-child status with educational statuses and mobility indicators, respectively. Two main findings were derived from the model results. First, as shown by the non-significant interaction terms in Model 3, the relative influence of origin and destination to subjective well-being did not differ significantly between only children and those with siblings. Second, as shown in Model 4, only children on downward trajectories were less satisfied with life (-0.18) than their counterparts with siblings. These findings partially validated the only-child hypothesis (H3a).

The diagonal intercepts, detailed in Part III of Table~\ref{tab:dmm_results}, show no distinct gradient in life satisfaction among non-mobile individuals across different educational levels, suggesting that in case of stagnation, subjective well-being was not linearly associated with education. Regarding the covariates, primary movers who were female, younger, Han-Chinese, married, or members of the Communist Party reported significantly higher life satisfaction, whereas those residing in urban areas or having economic relationships with parents were less satisfied with life.

In the second stage of the analysis, I stratified the samples by only-child status. The results of the DMMs on life satisfaction of only children are summarized in Table~\ref{tab:dmm_only_children}. The subsequent discussion focuses on only children's wellbeing outcomes of mobility.

The baseline model shows that their life satisfaction was influenced by both parents' (0.40) and their own (0.60) education. While the influence of parents' education appeared to be more pronounced for only children than for those with siblings, the difference was not statistically significant, as indicated by interactions terms in earlier models. In Model 2, the additional mobility indicators significantly improved the model fits, suggesting that mobility exerted independent effects on subjective well-being, net of statuses. However, these effects were asymmetric. Only downward mobility was significantly and negatively (-0.22) associated with life satisfaction of only children, whereas upward mobility had no significant impact. These findings, with regard to only children, validated the mobility hypotheses H2 and H2a and rejected the dissociative hypothesis (H2b).

To evaluate the gender disparity hypothesis (H3b), I interacted respondents' gender with educational statuses and mobility indicators in Models 3 and 4, respectively. None of the interaction terms was significant. Thus, I rejected hypothesis H3b and concluded that the relative influence of origin and destination on subjective well-being did not differ between sons and daughters from only-child families, nor did the negative effects of downward mobility vary by gender. Intergenerational educational mobility exerted much the same influence on sons and daughters from only-child families.

No distinct gradient in life satisfaction was observed among non-mobile only children across different educational levels. Sociodemographic factors such as being female, younger, married, or Party members were associated with higher life satisfaction. Urban residency and economic relationships with parents were related to lower life satisfaction.

In the final phase of the analysis, I expanded the framework to include parents from only-child families, examining how their life satisfaction was influenced by their children's educational mobility. These results are summarized in Table~\ref{tab:dmm_only_child_parents}.

As shown by the baseline model, parents' life satisfaction was simultaneously influenced by origin (0.78) and destination (0.22), with their own status being more influential. Model 2 incorporates additional dummies of upward and downward mobility. Results indicate that parents felt less satisfied with life (-0.14) when their child experienced downward mobility but were unaffected by upward mobility.

Subsequent models (Models 3 to 6) include interactions terms to explore variations by parents' and the child's gender. Results of these models show that the relative influence of origin and destination, as well as the detrimental effects of downward mobility, did not differ between fathers and mothers, nor did it vary by the child's gender. The educational attainment and mobility of the daughter were as impactful as those of the son for parents from only-child families, pointing to a convergence, if not a reversal, of son- and daughter-parent relationships.

In Models 7 and 8, I introduced three-way interactions to determine if these patterns hold for both fathers and mothers. Notably, in Model 8, the interaction term among upward mobility, mother, and daughter was significantly positive, suggesting gender differentials by parent-child pairwise dyad. Mothers were more satisfied with life if their daughter achieved upward mobility, whereas the son's mobility did not have such effects. Among all parents, mothers with an upwardly mobile daughter reported the highest life satisfaction. On the other hand, educational mobility of the only son and the only daughter had similar effects on their fathers’ subjective well-being. These findings challenge the notion that traditional son preferences have persisted in only-child families and are more strongly endorsed by mothers than fathers \parencite{liuBoysOnlychildrenGirls2006,sunAreMothersSons2017}. Instead, they provide preliminary evidence on a maternal preference for daughters. Hypotheses H3c and H4 were thus rejected.

Contrary to the lack of a distinct gradient in life satisfaction among non-mobile primary movers across the educational ladder, I observed an overall negative gradient in life satisfaction among parents with non-mobile only children. Specifically, the levels of life satisfaction were 2.91, 2.83, 2.71, and 3.03 for those with primary school or lower, middle school, high-school diploma, and college or higher degrees, respectively. This suggests a heightened dissatisfaction with status stagnation among higher-educated parents, except for those already at the top educational level. Among the covariates, being female, older, married, party members, and living in rural areas were positively associated with life satisfaction.

From the above findings, I draw three important conclusions. First, both parents' and adult children's educational statuses were significant determinants of subjective well-being, with a relatively greater influence from individuals' own education. Second, downward mobility adversely affected subjective well-being, a phenomenon exclusively observed in only-child families, whereas upward mobility had no significant impact. Third, the son- and daughter-parent contracts in only-child families have largely converged and, to some extent, shifted in favor of the daughter.
