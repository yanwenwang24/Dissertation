\section{Hypotheses}
\label{sec:ch2-hypotheses}

From the preceding theoretical frameworks, I formulate four sets of hypotheses concerning both primary movers and their parents.

First, the acculturation theory emphasizes the relative influence of educational statuses at the origin and destination. For analytical purposes, I purpose the acculturation hypothesis in its most strict form.

\begin{quote}
    \textbf{H1.} Subjective well-being is affected and only affected by origin and destination. Once these are accounted for, no additional mobility effects remain.
\end{quote}

Second, mobility may exert an independent effect on subjective wellbeing. The MDT suggests varying effects of mobility by direction, with downward mobility being particularly harmful \parencite{michalosMultipleDiscrepanciesTheory1985}. On the other hand, \citeauthor{sorokin1927social}'s (\citeyear{sorokin1927social}) dissociative theory posits disruptive effects of mobility in both directions. This leads to the mobility hypothesis.

\begin{quote}
    \textbf{H2.} Subjective well-being is affected by mobility, net of origin and destination.
\end{quote}

\begin{quote}
    \textbf{H2a.} Subjective well-being is positively associated with upward mobility and negatively related to downward mobility, with the latter having a larger absolute effect.
\end{quote}

\begin{quote}
    \textbf{H2b.} Subjective well-being is negatively associated with mobility in either direction.
\end{quote}

The third set of hypotheses focuses on the role of family structures in shaping the well-being outcomes of educational mobility. The only-child hypothesis compares only children with those with siblings. Drawing on traditional son preferences, the gender disparity hypotheses contrast only sons with only daughters, and parents of only son with those of only daughter, with differences between fathers and mothers.

\begin{quote}
    \textbf{H3a.} Among primary movers, the relative influence of origin on subjective well-being, as well as the effects of net mobility, is greater for only children than for those with siblings.
\end{quote}

\begin{quote}
    \textbf{H3b.} Among only children, the relative influence of destination on subjective well-being, as well as the effects of net mobility, is greater for sons than for daughters.
\end{quote}

\begin{quote}
    \textbf{H3c.} Among parents, the relative influence of destination (compared to origin), as well as the effects of net mobility, is greater for parents with an only son than for those with an only daughter. Differences by child's gender may be greater for mothers than for fathers.
\end{quote}

Lastly, acknowledging mothers' more intensive involvement in childcare, I propose:

\begin{quote}
    \textbf{H4.} Among parents, the relative influence of destination (compared to origin) on subjective well-being, as well as the effects of mobility, is greater for mothers than for fathers.
\end{quote}
