\section{Introduction}
\label{sec:ch2-introduction}

% Introduction
Intergenerational educational mobility, defined as the upward or downward movement across parents' (origin) and adult children's (destination) educational statuses, has garnered much research attention. Sociological studies on this topic can be broadly categorized into three strands. The first describes the trends and magnitudes of educational mobility across contexts (e.g., \cite{gruijtersTrendsEducationalMobility2019,songLongtermDeclineIntergenerational2020,torcheEducationalMobilityDeveloping2021,xieTrendsSocialMobility2022}). Fueled by global educational expansion, there has been a notable increase in upward educational mobility. The proportions of children having higher education than their parents peaked at 65\% in the 1950s cohorts in high-income countries and 50\% in the 1960s cohorts in developing regions, but both were followed by decades of stagnation and in some cases, decline \parencite{torcheEducationalMobilityDeveloping2021}. The second strand addresses the societal implications of mobility, particularly concerning social stratification and inequality (e.g., \cite{bianChineseSocialStratification2002,goldthorpeSocialMobilityClass1980,yeungHigherEducationExpansion2013}). Generally agreed upon, societies with less rigid intergenerational transmission of status are more equal. The third strand, marking a transition from objective to subjective, from macro to micro, focuses on individuals' subjective experience of ascending or descending the social ladder, with which this study aligns (e.g., \cite{dhooreSocialMobilityLife2019,kwonImpactIntergenerationalMobility2022,schuckDoesIntergenerationalEducational2018,zangFrustratedAchieversSatisfied2016}).

The well-being consequences of educational mobility have been extensively theorized, with a focus on \textit{primary movers}—individuals who themselves move up or down the educational ladder. \textcite{blauSocialMobilityInterpersonal1956} conceptualized mobility as an acculturation process, wherein primary movers, neither fully integrated into the destination class nor completely constrained by their origin, were influenced by the normative beliefs, attitudes, and behaviors at both origin and destination. Other scholars emphasize the symbolic significance of mobility \textit{per se}, that is, the difference between origin and destination, which may independently affect individuals' well-being via the success or failure of fulfilling personal, familial, and societal expectations, or social isolation from both positions \parencite{michalosMultipleDiscrepanciesTheory1985,sorokin1927social,sorokin1959social}. However, empirical evaluation of these theories has long been hindered by the lack of methodological tools to disentangle the linear dependency among origin, destination, and mobility, until the introduction of the Diagonal Mobility Model (DMM) \parencite{sobelDiagonalMobilityModels1981}. The DMM, which will be reviewed in detail in the methods section, treats non-mobile individuals as the core of their respective social positions and allows for the isolation of net mobility from the respective influence of origin and destination. Using this approach, studies have found that primary movers' subjective well-being is influenced by both their parents' and their own statuses, with the latter of greater influence across contexts \parencite{dhooreSocialMobilityLife2019,kwonImpactIntergenerationalMobility2022,schuckDoesIntergenerationalEducational2018,zangFrustratedAchieversSatisfied2016}. A few exceptions notwithstanding \parencite{schuckDoesIntergenerationalEducational2018}, most studies did not find any additional mobility effects once statuses have been accounted for.

In recent years, the linked lives perspective has gained prominence in family research \parencite{bengtsonLifecoursePerspectiveAgeing2005,elderFamiliesLivesDevelopments1987,elderFamilyHistoryLife1977,macmillanFamiliesLifeCourse2005}. It recognizes that an individual's life is embedded in the family and intertwined with the lives of other family members. Given the strong connections between parents and children throughout the life course, the well-being ramifications of educational mobility will likely reverberate across generations, extending beyond primary movers to their parents. This is especially pertinent in the Chinese context, where education is viewed as a family asset, upward mobility is pursued as a family project, and parents harbor high aspiration for children's success, along with deep-seated fear of falling downward \parencite{chenEducationFeverChina2021,guSacrificeIndebtednessIntergenerational2022,guSacrificeIndebtednessIntergenerational2022,mengWhenAnxiousMothers2020}. While studies have shown that older parents' psychological well-being is strongly influenced by various life events and conditions of their children, including health, marriage, and education \parencite{friedmanSchoolingOffspringSurvival2014,maDoesAdultChildrens2022,zhangChildrensTransitionsAdulthood2023}, the association between children's educational mobility and parents' well-being remains unexplored.

By including parents and contextualizing educational mobility in intergenerational relationships, the study further draws attention to family structures and their roles in shaping both generations' subjective experience of mobility. Particularly noteworthy is the rise of only-child families as a result of the historical One-Child Policy implemented between 1979 and 2015 \parencite{settlesOneChildPolicyIts2013}. In such families, the only child, regardless of gender, enjoys exclusive access to educational resources and represents the family's sole opportunity for upward intergenerational mobility \parencite{falboQuantitativeReviewOnly1986}. This unique context allows me to address two underexplored research gaps. First, regarding the primary movers, it remains uncertain whether educational mobility is of greater consequence, in either positive or negative way, to the well-being of only children compared to those with siblings. Second, the growing prevalent of only-child families has improved gender equality, yet it remains too optimistic to conclude a thorough elimination of traditional son preferences by only-child status without an empirical comparison between only-son and only-daughter families \parencite{fongChinasOneChildPolicy2002,liuBoysOnlychildrenGirls2006,wuFertilityDeclineWomens2014}. Furthermore, the extent to which son preferences are endorsed—in this case, undermined by the only-child status—may vary between fathers and mothers \parencite{sunAreMothersSons2017}, and the intergenerational transmission of status and gender ideology are distinct by parent-child pairwise dyad \parencite{huGenderEducationExpansion2023}. These nuanced gender dynamics within only-child families have largely remained unexplored with regard to the well-being outcomes of mobility. This study addresses these issues by analyzing whether only children's subjective experiences of educational mobility vary by gender and differ from those with siblings, and whether those of fathers and mothers differ based on the child's gender.

This study focuses on the Chinese context where educational mobility is widely regarded as the primary route to broader social mobility, and only-child families have become predominant due to the historical One-Child Policy \parencite{settlesOneChildPolicyIts2013}. I utilize the DMM to analyze a representative sample of 1475 adult children and 3808 parents from only-child families, as well as 16,385 primary movers unrestricted by the only-child status, selected from the China Family Panel Studies (CFPS) in 2010. I assess the extent to which educational mobility, as well as educational statuses at origin and destination, affects the subjective well-being of both primary movers and their parents. Additionally, I examine if such influence varies by gender and only-child status.

The article is organized as follows. I begin by reviewing the theoretical framework and raising hypotheses, followed by detailing the data, measures, and analytical strategies. I then present descriptive statistics and empirical findings and conclude by summarizing the results and discussing their limitations and implications for future mobility research.
